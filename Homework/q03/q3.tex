\documentclass[12pt]{article}
\usepackage[margin=1in]{geometry}
\usepackage{fancyhdr}
\pagestyle{fancy}
\setlength{\parindent}{0em}
\setlength{\parskip}{0.5em}

\lhead{25 January 2016}
\chead{ASTR 535 - Homework 03}
\rhead{Laurel Farris}

\begin{document}

\bibliographystyle{plain}

\section*{Homework 3}
\subsubsection*{What is the solar constant (definition and value)?
    What is the surface brightness of the Sun as seen from Earth?
    What would the solar constant and surface brightness of the Sun
    be as seen from Jupiter?}

    The solar constant is the bolometric flux received from the sun at a
    distance of 1 astronomical unit (AU)\cite{kuhn}.
    The solar luminosity is 3.9$\times10^{33}$ erg s$^{-1}$, so at a distance of
    1.5$\times10^{13}$ cm, the solar constant is equal to
    1.38$\times10^6$ erg s$^{-1}$ cm$^{-2}$.
    The surface brightness of the sun as seen from Earth is
    2.04 $\times10^{10}$ erg s$^{-1}$ cm$^{-2}$ sterradian$^{-1}$.
    This was calculated by dividing the flux, or solar constant, by
    the solid angle subtended by the sun:
    $ \Omega \approx \frac{\pi R_{odot}^2}{d^2} $

    From Jupiter, which is about 5 AU (7.5$\times10^{13}$ cm) from the Sun,
    the solar constant is about 5.52$\times10^4$ erg s$^{-1}$ cm$^{-2}$.
    The surface brightness of the Sun from Jupiter
    is the same as it is from Earth, as
    surface brightness is independent of distance at these scales.

    (2/12/16 Post-grading notes: 
    Original formula for solid angle:
    $\theta$ = [2arctan($\frac{R_{\odot}}{1\ \textrm{AU}}$)]$^2$.
    I got this by calculating the one-dimensional angle subtended by
    the sun, and then squaring it. The factor of 2 comes in because I
    drew a right triangle, using the sun's radius as 'opposite' and
    one AU as 'adjacent' when calculating the arctangent of the angle.
    So that was the angle subtending by only half the sun. I don't
    bother with small-angle approximations when the trig is pretty
    straightforward. Since I got the same order of magnitude, I guess
    this method is sort of close... but I used
    $ \Omega \approx \frac{\pi R_{odot}^2}{d^2} $
    the second time).

\bibliography{reffile}

\end{document}
