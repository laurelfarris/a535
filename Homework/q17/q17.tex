\documentclass[12pt]{article}
\usepackage[margin=1in]{geometry}
\setlength{\parindent}{0em}
\setlength{\parskip}{0.5em}
\usepackage{fancyhdr}
\pagestyle{fancy}
\usepackage{color}
\usepackage{hyperref}

\lhead{Laurel Farris}
\chead{ASTR 535 - Homework 17}
\rhead{26 February 2016}

\begin{document}

\paragraph{(2/26, 30 minutes) The spectroscopic plates for the Sloan
Digital Sky Survey cover a circle about 3 degrees in diameter on the
sky, and 2 arcsecond (diameter) holes are drilled in the plate at the
location of different objects in the sky. Given two stars that have
the same RA and declinations 0 and 3 degrees, for an observation taken
on the meridian, will it be acceptable to use a plate that that has
holes that are drilled exactly 3 degree apart? What is the optimal
separation for the holes? Do you think the optimal separation will be
the same if the field was observed at HA=3 hours?}



\end{document}
