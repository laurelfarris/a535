\documentclass[12pt]{article}
\usepackage{hyperref}
\usepackage{color}

\begin{document}
\section*{Homework 2}
\subsection*{Look at the APO website at
\textcolor{blue}{\url{http://www.apo.nmsu.edu}}}

\subsubsection*{Look at the observing schedule and some observing night logs.
    What are the NMSU programs this quarter?}

    Under 1Q2016 (the first quarter for this year), there are eight different
    observing runs listed, each of which include the dates, start/end times,
    the instrument to be used, and the moon brightness (0 for new moon,
    100 for full moon?).
    The instruments to be used, as written on the site, are
    `V', `DRE', `RS-custom', `D-B1200+R1200', `N', `E', and `R-MSSSO?'.
    There is no instrument listed under the third observing run.
    The second run is for our training in March, using the DRE\@.
    I am currently trying to find out what this is.

\subsubsection*{Check out the 10 micron all-sky camera images at night. What are
    the images and graphs showing, and what do they tell you?}

    I clicked the link to the IR All-Sky Camera Image (which I assume is the
    same thing since 10 $\mu$ is in the IR). The two plots show the sky
    standard deviation and the sky level, both as functions of UT Time.
    They both describe the `sky history'.
    I assume the sky counts are used for data reduction.
    The images appear to be centered on either just the sky, or a very
    unresolved star. I may be looking at the wrong images.

\subsubsection*{Look through the instruments pages, and at anything else that might
    interest you. If anything stimulates a question, write it down.}



\subsubsection*{Submit housing request for 3/25--3/27 (3 nights).}

    Success!

\end{document}
