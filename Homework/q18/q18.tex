\documentclass[12pt]{article}
\usepackage[margin=1in]{geometry}
\setlength{\parindent}{0em}
\setlength{\parskip}{0.5em}
\usepackage{fancyhdr}
\pagestyle{fancy}
\usepackage{color}
\usepackage{hyperref}

\lhead{Laurel Farris}
\chead{ASTR 535 - Homework 18}
\rhead{26 February 2016}

\begin{document}

\paragraph{(2/26, 60 minutes) As we'll discuss, seeing causes the
observed shape of a star to \emph{roughly} be a (2D) Gaussian. For a star
that has a (2D) Gaussian profile, calculate the radii at which 50\%,
95\%, and 98\% of the total light is enclosed as a function of the FWHM
of the star. Given your calculations, what is a reasonable choice for
an aperture radius to choose to count a significant fraction of the
total light from a star? If the seeing (FWHM) varies by 10-20\% from
exposure to exposure, what radius might be required to insure that the
fraction of the total flux from exposure to exposure varies by less
than 1\%?}


\end{document}
