\documentclass[12pt]{article}
\usepackage{color}
\usepackage{hyperref} % \url{link} or \hrf{link}{text}
\usepackage{titlesec}
\usepackage{amsmath}
\titleformat{\section}%
  [hang]% <shape>
  {\normalfont\bfseries\Large}% <format>
  {}% <label>
  {0pt}% <sep>
  {}% <before code>
  \renewcommand{\thesection}{}% Remove section references...
  \renewcommand{\thesubsection}{\arabic{subsection}}%... from subsections
  \renewcommand{\thesubsubsection}{\arabic{subsubsection}}%... from subsections

\addtolength{\oddsidemargin}{-0.875in}
\addtolength{\evensidemargin}{-0.875in}
\addtolength{\textwidth}{1.75in}
\addtolength{\topmargin}{-0.875in}
\addtolength{\textheight}{1.75in}

\begin{document}

%\bibliographystyle{plain}

\section*{Homework 7}
\date{01/29/2016}
\subsection*{Astronomers use several different units to measure
brightnesses: fluxes per unit wavelength ($F_{\lambda}$),
fluxes per unit frequency ($F_{\nu}$), and magnitudes, both
integrated and per unit wavelength and frequency (m = -2.5logF).
Since magnitudes are logarithimic units, differences in magnitude
correspond to ratios in fluxes.}

\subsubsection{A Jansky is a unit used to measure flux density,
most often in the radio; one Jansky is 10$^{-26}$ W m$^{-2}$ Hz$^{-1}$
(note that this is an F$_{\nu}$ quantity).
How bright is Vega at 5500 \AA{} in Janskys, using the fact that the
flux density of Vega at 5500 \AA{}
is 3.6 $\times 10^{-9}$ erg cm$^{-2}$ s$^{-1}$ \AA{}$^{-1}$
(note that this is an F$_{\lambda}$ quantity)?}

In cgs units, one Jansky [Jy] is 10$^{-23}$
erg s$^{-1}$ cm$^{-2}$ Hz$^{-1}$. 
The flux of Vega at 5500\AA{} is 3.63$\times$10$^{-20}$
erg s$^{-1}$ cm$^{-2}$ Hz$^{-1}$.
\begin{align*}
    F_{Jy} &= \frac{3.63\times10^{-20}}{10^{-23}}\\
    &= 363.0
\end{align*}
So the flux of Vega is 363 Jy.


\subsubsection{If a star has a flux density of
3.6$\times$10$^{-9}$ erg cm$^{-2}$ s$^{-1}$ \AA{}$^{-1}$
at 8500 \AA{} how bright is it in Janskys?}
Using the function `convert\_flux' from the `conversions.py' module
from q5,
$F_{\lambda}$ was converted to $F_{\nu}$.
Using the same calculation
as in question 1, the flux of the star came out to be
$\sim$ 8.676$\times$10$^{-6}$ Jy.

\subsubsection{If a star has a flux density of 7.2$\times$10$^{-14}$
erg cm$^{-2}$ s$^{-1}$ \AA{}$^{-1}$ at 5500 \AA{},
how much fainter is it than Vega in magnitudes? }
\begin{align*}
    m_* - m_{Vega} &= -2.5 \log \frac{F_*}{F_{Vega}}\\
    &= -2.5 \log \frac{7.2\times10^{-14}}{3.6\times10^{-9}}\\
    &= 11.747
\end{align*}
So the difference in magnitude between the star and Vega is over 11
magnitudes, in other words, the star is about 50,000 times fainter
than Vega.

%\bibliography{reffile}

\end{document}
