\documentclass[12pt]{article}
\usepackage{color}
\usepackage{hyperref}
\usepackage{amsmath}

\title{ASTR 535 Lecture Notes}
\author{Jon Holtzman}
\date{Spring 2016}

\addtolength{\oddsidemargin}{-0.875in}
\addtolength{\evensidemargin}{-0.875in}
\addtolength{\textwidth}{1.75in}
\addtolength{\topmargin}{-0.875in}
\addtolength{\textheight}{1.75in}


\begin{document}
\maketitle

\noindent course website: \textcolor{blue}
{\url{http://astronomy.nmsu.edu/holtz/a535}}\\

\noindent \textcolor{magenta}{Friday, January 22}
\section*{Properties of light, magnitudes, errors, and error analysis}

\subsection*{Light}
Wavelength regimes:
\begin{itemize}
    \item gamma rays
    \item x-rays
    \item ultraviolent (UV)
        \begin{itemize}
            \item near: 900--3500 \AA{}
            \item far: 100--900 \AA{}
        \end{itemize}
        The 900 \AA{} break is because of the Lyman limit at 912 \AA{}.
        This is where neutral hydrogen is ionized, so the universe is largely
        opaque to wavelengths shorter than this.
    \item visual (V): 4000--7000 \AA{} (note that `V' is different from `optical',
        which is slightly broader: 3500--10000 \AA{}. The 3500 \AA{} cutoff
        is due to the Earth's atmosphere being opaque to wavelengths shorter
        than this).
    \item IR
        \begin{itemize}
            \item near: 1--5 $\mu$ (1--10 $\mu$ in online notes)
            \item mid: (10--100 $\mu$)
            \item far: 5--100 $\mu$ (100--1000 $\mu$)
        \end{itemize}
    \item sub-mm 500--1000 $\mu$
    \item microwave
    \item radio
\end{itemize}
Quantities of light:
\begin{itemize}
    \item Intensity $I(\theta,\phi)$ [erg $s^{-1}$ $\Omega^{-1}$ $\nu^{-1}$]
        Encapsulates $direction$ light is coming from.
        Also known as radiance.
    \item Surface Brightness (SB)
        [erg s$^{-1}$ cm$^{-2}$ $\nu^{-1}$ sterradian$^{-1}$]
        amount of energy $received$ in a unit surface
        element per unit time per unit frequency (or wavelength) from a unit
        solid angle in the direction ($\theta,\phi$), where $\theta$ is the angle
        away from the normal to the surface element, and $\phi$ is the azimuthal
        angle. SB is independent of distance unless considering cosmological
        scales, where the curvature of spacetime has an effect.
    \item Flux (F): amount of energy passing through a unit surface element
        in all directions, defined by
        \begin{equation}
            F_{\nu} = \int I_{\nu}\cos(\theta)\textrm{d}\Omega
        \end{equation}
        where d$\Omega$ is the solid angle element, and the integration is
        over the entire solid angle. The $\cos(\theta)$ factor is important
        for, e.g., ISM where light is coming from all directions, but for
        tiny objects, $\theta$ is negligibally small and can be dropped.
        Integrates over $all$ directions.
        Also known as irradiance.
    \item Luminosity (L): [erg s$^{-1}$]
        $intrinsic$ energy emitted by the source per
        second ($\sim$ power). For an isotropically emitting source,
        \begin{equation}
            L = 4 \pi d^2 F
        \end{equation}
        where $d$ = distance to source. Also known as radiant flux.
\end{itemize}
What to measure for sources:
\begin{itemize}
    \item Resolved: directly measure surface brightness (intensity)
        distribution on the sky, usually over some bandpass or wavelength
        interval.
    \item Unresolved: measure the flux. Diffraction is the reason stellar
        surfaces cannot be resolved. Because of this, we cannot measure
        SB, so we measure flux, integrated over the entire object.
\end{itemize}
Note that Luminosity can only be calculated if the distance is known.\\

\noindent Questions:
\begin{itemize}
    \item What are the dimensions of the three quantities: luminosity,
        surface brightness (intensity), and flux?
    \item How do the three quantities depend on distance to the source?
    \item To what quantity is apparent magnitude of a star related?
    \item To what quantity is the absolute magnitude related?
\end{itemize}
Amount of light emitted is a function of wavelength, so we are often interested
in e.g.\ flux per unit wavelength or frequency, also known as $specific$
flux. Using $\lambda=\frac{c}{\nu} \rightarrow
\frac{\textrm{d}\lambda}{\textrm{d}\nu} = \frac{-c}{\nu^2}$
\begin{align*}
    \int F_{\nu} \textrm{d} \nu &= \int F_{\lambda} \textrm{d} \lambda\\
    F_{\nu} \textrm{d} \nu &= F_{\lambda} \textrm{d} \lambda\\
    F_{\nu} &= F_{\lambda} \frac{\textrm{d} \lambda}{\textrm{d}\nu}\\
    &= -F_{\lambda} \frac{c}{\nu^2}\\
    &= -F_{\lambda} \frac{\lambda^2}{c}\\
\end{align*}
Note that a constant $F_{\lambda}$ implies a $non$-constant $F_{\nu}$
and vice versa. Depending on where you are, a constant chunk of 1 Hz is
not the same wavelength range, depending on where you are.\\

\noindent Units: often cgs, magnitudes, Jansky (a flux density unit
corresponding to 10$^{-26}$ W m$^{-2}$ Hz$^{-1}$)\\\\
\noindent There are often variations in terminology\\\\
\noindent Terminology of measurements:
\begin{itemize}
    \item photometry (broad-band flux measurement) SB or flux, integrated
        over some wavelength range.
    \item spectroscopy (relative measurement of fluxes at different wavelengths)
        $f(\lambda)$
    \item Spectrophotometry (absolute measuremnet of fluxes at different wavelengths)
        $f(\lambda)$
    \item astrometry: concerned with positions of observed flux, not brightness,
        but direction.
    \item morphology: intensity as a function of position;
        often, absolute measurements are unimportant. Deals with $resolved$
        objects, intensity as function of position.
\end{itemize}
Generally, measure $flux$ with photometry, and flux $density$ with spectroscopy
(down to the resolution of the spectrograph). In practice, with most detectors,
we measure photon flux [photons cm$^{-2}$ s$^{-1}$] with a photon counting device,
rather than energy flux (which is done with bolometers).
The monochromatic photon flux is given by the energy flux ($F_{\lambda}$)
divided by the energy per photon ($E_{photon} = \frac{hc}{\lambda}$),
or
\begin{equation*}
    \textrm{photon\ flux} = \int F_{\lambda} \frac{\lambda}{hc} \textrm{d} \lambda
\end{equation*}

\subsection*{Magnitudes and photometric systems}
Magnitudes are related to flux (and SB and L) by:
\begin{equation*}
    m_1 - m_2 = -2.5 \log \frac{b_1}{b_2}
\end{equation*}
or for a single object:
\begin{align*}
    m &= -2.5 \log \frac{F}{F_0}\\
      &= -2.5 \log F + 2.5 \log F_0
\end{align*}
where the coefficient of proportionality, $F_0$, depends on the definition
of photometric system; the quantity $-2.5 \log F_0$ may be referred to as
the photometric system zeropoint. Note that this relationship holds
regardless of what photometric system you are using. Inverting, one gets:
\begin{equation*}
    F = F_0 \times 10^{-0.4\textrm{m}}
\end{equation*}
Just as fluxes can be represented in magnitude units, flux densities can be
specified by monochromatic magnitudes:
\begin{equation*}
    F_{\lambda} = F_0 (\lambda) \times 10^{-0.4 \textrm{m}(\lambda)}
\end{equation*}
although spectra are more often given in flux units than in magnitude units.
Note that it is possible that $F_0$ is a function of wavelength.\\

\noindent Since magnitudes are logarithmic, the $difference$ between
magnitudes corresponds to a ratio of fluxes; ratios of magnitudes are
generally unphysical. If one is just doing relative measurements of
brightness between objects, this can be done without knowledge of $F_0$
(or, equivalently, the system zeropoint); objects that differ in brightness
by $\Delta$M have the same ratio of brightness (10$^{-0.4 \Delta M}$)
regardless of what photometric system they are in. The photometric system
definitions and zeropoints are only needed when converting between calibrated
magnitudes and fluxes. Note that this means that if one references the
brightness of one object relative to that of another, a magnitude system
can be set up relative the brightness of the reference source. However, the
utility of a system when doing astrophysics generally requires an
understanding of the actual fluxes.\\

\noindent \textcolor{magenta}{Monday, January 25}\\

\noindent There are three main types of magnitude systems in use in astronomy.
We start by describing the two simpler ones: the STMAG and the ABNU mag system.
In these simple systems, the reference flux is just a constant value in
$F_{\lambda}$ or $F_{\nu}$. However, these are not always the most widely used
systems in astronomy, because no natural source exists with a flat spectrum.\\

\noindent The STMAG (Space Telescope Mag) is relative to a source of
constant $F_{\lambda}$. In this system, $F_{0,\lambda} = 3.60 \times 10^{-9}$
erg cm$^{-2}$ s$^{-1}$ \AA{}$^{-1}$, which is the flux of Vega at
5500 \AA{}; hence a star of Vega's brightness at 5500 \AA{} is defined to
have m=0. Alternatively, we can write
\begin{equation*}
    m_{STMAG} = -2.5 \log F_{\lambda} - 21.1
\end{equation*}
for $F_{\lambda}$ in cgs units.\\

\noindent In the ABNU system, things are defined relative to a source
of constant $F_{\nu}$ and we have
\begin{equation*}
    F_{0,\nu} = 3.63 \times 10^{-20} erg\ s^{-1}\ cm^{-2}\
    Hz^{-1}\ 10^{-0.4 \textrm{m}_{\nu}}
\end{equation*}
or
\begin{equation*}
    m_{ABNU} = -2.5 \log F_{nu} - 48.6
\end{equation*}
for $F_{\nu}$ in cgs units. Again, the constant comes from the flux of
Vega.\\

\noindent Magnitudes usually refer to the flux integrated over a
spectral bandpass. In this case, $F$ and $F_0$ refer to fluxes
integrated over the bandpass. The STMAG and ABMAG integrated systems
are defined relative to sources of constant $F_{\lambda}$ and
$F_{\nu}$ systems, respectively.
\begin{align*}
    m_{STMAG} &= -2.5 \log \frac{\int F_{\lambda} \lambda
    \textrm{d}\lambda}{\int3.6\times10^{-9}\lambda\textrm{d}\lambda}\\
    m_{ABNU} &= -2.5 \log \frac{\int F_{\nu}/ \nu
    \textrm{d}\nu}{\int 3.6 \times 10^{-20}/ \nu\textrm{d}\nu}
\end{align*}
These should have the same value at 5500 \AA{}.
The factors of $\lambda$ and $\nu$ come from the conversion factor
$hc/\lambda$, where the constants, $h$ and $c$, cancel out in the
fractions in each equation.\\

\noindent Note that these systems differ by more than a constant,
because one is defined by units of $F_{\lambda}$ and the other by
$F_{\nu}$, so the difference betwen the systems is a function of
wavelength. They are defined to be the same at 5500 \AA{}. (Question:
what's the relations between $m_{STMAG}$ and $m_{ABNU}$?)\\

\noindent Note also that, using magnitudes, the measured magnitude is
nearly independent of bandpass width, so a broader bandpass does not
imply a brighter (smaller) magnitude, which is not the case for
fluxes. The reference is being integrated as well, so they cancel.\\

\noindent The standard UBVRI broadband photometric system, as well as
several other magnitude systems, however, are not defined for a
constant (flat) $F_{\lambda}$ or $F_{\nu}$ spectrum; rather, they are defined
relative to the spectrum of an A0V star. Most systems are defined (or
at least were originally) to have the magnitude of Vega be zero in all
bandpasses (VEGAMAGS); if you ever get into this in detail, note that
this is not exactly true for the UBVRI system.\\

\noindent For the broadband UBVRI system, we have
\begin{equation*}
    m_{UBVRI} \approx -2.5 \log
    \frac{\int_{UBVRI}F_{\lambda}(object)\lambda\textrm{d}\lambda}
    {\int_{UBVRI}F_{\lambda}(Vega)\lambda\textrm{d}\lambda}
\end{equation*}
(as above, the factor of $\lambda$ comes in for photon counting
detectors). This gives the magnitude in U, B, V, R, \emph{or} I,
by integrating over that same bandpass.
The UBVRI filter set had overlapping bandpasses, so
there was a switch to interference filter: the ugriz system used
by SDSS (explained below... I think).\\

\noindent Here is a
\href{http://astronomy.nmsu.edu/holtz/a535/html/diagrams/a535/mag.htm}
{\textcolor{blue}{plot}}
to demonstrate the difference between the different systems.\\

\noindent While it seems that STMAG and ABNU systems are more
straightforward, in practice it is difficult to measure absolute
fluxes, and much easier to measure relative fluxes between objects.
Hence, historically observations were tied to observations of Vega (or
to stars which themselves were tied to Vega), so VEGAMAGs made sense,
and the issue of determining physical fluxes boiled down to measuring
the physical flux of Vega. Today, in some cases, it may be more
accurate to measure the absolute throughput of an instrumental system,
and using STMAG or ABNU makes more sense.

\subsubsection*{Colors}
Working in magnitudes, the difference in magnitudes between different
bandpasses (called the color index, or simply, color) is related to
the flux ratio between the bandpasses, i.e., the color.
In the UBVRI
system, the \emph{difference between magnitudes}, e.g. B-V,
gives the ratio of the fluxes in different bandpasses
\emph{relative to the ratio of the fluxes of
an A0V star in the different bandpasses (for VEGAMAG)}.
Note the typical colors of astronomical objects,
which are different for the different photometric systems.\\
\emph{A} stars have color 0, have same SED as Vega. \emph{O} stars
have color less than 0, while coolor stars have color greater than 0.
In sloan system, have  g-r (ugriz). g-r=0 indicates constant
$F_{\nu}$.
\begin{equation*}
    m = -2.5\log\frac{\int_B F_{\lambda}\textrm{d}\lambda}
    {\int_V F_{\lambda}\textrm{d}\lambda}
\end{equation*}
if B-V=0, then (B/V)$_{object}$ is the same as (B/V)$_{Vega}$.\\

\noindent What would the flux be from an object with some magnitude,
x? Need this to know how much observing time I need. E.g., convert the
spectrum of an elliptical galaxy to color; if you know F in one
filter, you can get F in another filter.

\noindent Which is closer to the UBVRI system, STMAG or ABNU?\\
\noindent What would typical colors be in a STMAG or ABNU system?

\newpage
\noindent \textcolor{magenta}{Wednesday, January 27}\\

\subsubsection*{UBVRI magnitudes-flux conversion}
To convert Vega-based magnitudes to fluxes, look up the flux of Vega
at the center of the passband; however, if the spectrum of the object
differes from that of Vega, this won't be perfectly accurate. Given
UBVRI magnitudes of an object in the desired band, filter profiles
(e.g. Bessell 1990, PASP 102,1181), and absolute spectrophotometry of
Vega (e.g., \href{http://adsabs.harvard.edu/abs/2004AJ....127.3508B}
{Bohlin \& Gilliland 2004, AJ 127, 3508}, one can determine
the flux.\\

\noindent If one wanted to estimate the flux of some object in
arbitrary bandpass given just the V magnitude of an object (a common
situation used when trying to predict exposures times, see below),
this can be done if an estimate of the spectral energy distribution
(SED) can be made; given the filter profiles, one can compute the
integral of the SED over the V bandpass, determine the scaling by
comparing with the integral of the Vega spectrum over the same
bandpass, then use the normalized SED to compute the flux in any
desired bandpass. Some possibly useful references for SEDs are:
Bruzual, Persson, Gunn, \& Stryker; Hunter, Christian, \& Jacoby;
Kurucz).\\

\noindent Things are certainly simpler in the ABNU or STMAG system, and
there has been some movement in this direction: the STScI gives STMAG
calibrations for HST instruments, and the SDSS photometric system is
close to an ABNU system.\\

\noindent Note, however, that even when the systems are conceptually
well defined, determining the absolute calibration of any photometric
system is very difficult in reality, and determining absolute fluxes
to the 1\% level is very challenging.\\

\noindent As a separate note on magnitudes themselves, note that some
people, in particular, the SDSS, have adopted a modified type of
magnitudes, called asinh magnitudes, which behave like normal (also
known as Pogson) magnitude for brighter objects, but have different
behavior for very faint objects (near the detection threshold); see
\href{http://adsabs.harvard.edu/abs/1999AJ....118.1406L}
{Lupton, Gunn, \& Szalay 1999 AJ 118, 1406} for details.

\subsection*{Observed fluxes and the count equation}
What if you are measuring flux with an actual instrument, i.e.
counting photons? The intrinsic photon flux from the source is not
trivial to determine from the number of photons that you count. To get
the number of photons that you count in an observation, you need to
take into account the area of your photon collector (telescope),
photon losses and gains from the Earth's atmosphere (which changes
with conditions), and the efficiency of your collection/detection
apparatus (which can change with time). Generally, the astronomical
signal (which might be a flux or a surface brightness, depending on
whether the object is resolved) can be written
\begin{equation*}
    S = Tt \int \frac{F_{\lambda}}{\frac{hc}{\lambda}}q_{\lambda}
    a_{\lambda}\textrm{d}\lambda \equiv TtS'
\end{equation*}
where $S$ is the observed photon flux (the ``signal"), $T$ is the
telescope collecting area, $t$ is the integration time, $a_{\lambda}$
is the atmospheric transmission (more later) and
$q_{\lambda}$ is the system efficiency (which includes
telescope, filters, optics, detector, etc.); $S'$ is an observed flux
rate, i.e. with all of the real details of the observing system
included. I refer to this as the \emph{count equation}.\\

\noindent Usually, however, one doesn't use this information to go
backward from $S$ to $F_{\lambda}$ because it is very
difficult to measure all of the terms precisely, and some of them
(e.g. $a$, and perhaps $q$) are time-variable; $a$ is also spatially
variable. Instead, most observations are performed differentially to a
set of other stars of known brightness. If the stars of known
brightness are observed in the same observation, then the atmospheric
term is (approximately) the same for all stars; this is known as
\emph{differential photometry}. From the photon flux of the object with known
brightness, one could determine an ``exposure efficiency'' for this
exposure. Equivalently, and more commonly, one can calculate an
\emph{instrumental magnitude}:
\begin{equation*}
    m = -2.5 \log \frac{S}{t}
\end{equation*}
(i.e., normalize by the exposure time to get counts/sec, although this
is not strictly necessary) and then determine the \emph{zeropoint} that needs
to be added to give the calibrated magnitude ($M$, make sure you
recognize that this is still an apparent magnitude):
\begin{equation*}
    M = m + z
\end{equation*}
Note that in the real world, one has to also consider possible
differences between a given experimental setup and the setup used to
measure the reference brightnesses, so this is only a first
approximation (i.e., the zeropoint may be different for different
stars with different spectral properties)! If using instrumental mags
including exposure time normalization, the zeropoint gives the
magnitude of a star that will give 1 count/second.\\

\noindent If there are no stars of known brightness in the same
observation, then calibration must be done against stars in other
observations. This then requires that the different effects of the
Earth's atmosphere in different locations in the sky be accounted for.
This is known as all-sky, or absolute, photometry. To do this requires
that the sky is ``well-behaved", i.e. one can accurately predict the
atmospheric throughput as a function of position. This requires that
there be no clouds, i.e. photometric weather. Differential photometry
can be done in non-photometric weather, hence it is much simpler! Of
course, it is always possible to obtain differential photometry and
then go back later and obtain absolute photometry of the reference
stars. It is also common to stop with differential photometry if one
is studying variable objects, i.e. where one is just interested in the
change in brightness of an object, not the absolute flux level.\\

\noindent Of course, at some point, someone needs to figure out what
the fluxes of the calibrating stars really are, and this requires
understanding all of the terms in the count equation. It is
challenging, and often, absolute calibration of a system is uncertain
to a couple of percent.\\

\noindent While the count equation isn't usually used for calibration,
it is very commonly used for computing the approximate number of
photons you will receive from a given source in a given amount of time
for a given observational setup. This number is critical to know in
order to estimate your expected errors and exposure times in observing
proposals, observing runs, etc. Understanding errors in absolutely
critical in all sciences, and maybe even more so in astronomy, where
objects are faint, photons are scarce, and errors are not at all
insignificant. The count equation provides the basis for exposure time
calculator (ETC) programs, because it gives an expectation of the
number of photons that will be received by a given instrument as a
function of exposure time. As we will see shortly, this provides the
information we need to calculate the uncertainty in the measurement as
a function of exposure time.

\subsection*{Errors in photon rates}
For a given rate of emitted photons, there's a probability function
which gives the number of photons we detect, even assuming 100\%
detection efficiency, because of statistical uncertainties. In
addition, there may also be instrumental uncertainties. Consequently,
we now turn to the concepts of probability distributions, with
particular interest in the distribution which applies to the detection
of photons.\\

\noindent \emph{Distributions and characteristics thereof}
\begin{itemize}
    \item concept of a distribution: define $p(x)dx$ as probability of
    event occuring in $x + dx$:
    \begin{equation*}
        \int p(x)\textrm{d}x = 1
    \end{equation*}
\end{itemize}
Some definitions relating to values which characterize a distribution:
\begin{align*}
    mean \equiv \mu &= \int xp(x)\textrm{d}x \\
    variance \equiv \sigma^2 &= \int (x-\mu)^2 p(x)\textrm{d}x \\
    standard\ deviation \equiv \sigma &= \sqrt{variance}
\end{align*}

\subsubsection*{Noise equation: how do we predict expected errors?}
\subsubsection*{Error propagation}
\subsubsection*{Determining sample parameters: averaging measurements}
\subsubsection*{Random errors vs systematic errors}

\end{document}














