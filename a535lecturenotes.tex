\documentclass[12pt]{article}
\usepackage{color}
\usepackage{hyperref}

\title{ASTR 535 Lecture Notes}
\author{Jon Holtzman}
\date{Spring 2016}

\addtolength{\oddsidemargin}{-0.875in}
\addtolength{\evensidemargin}{-0.875in}
\addtolength{\textwidth}{1.75in}
\addtolength{\topmargin}{-0.875in}
\addtolength{\textheight}{1.75in}


\begin{document}
\maketitle

\noindent course website: \textcolor{blue}{\url{http://astronomy.nmsu.edu/holtz/a535}}

\section*{Properties of light, magnitudes, errors, and error analysis}

\subsection*{Light}
Wavelength regimes:
\begin{itemize}
    \item gamma rays
    \item x-rays
    \item ultraviolent (UV)
        \begin{itemize}
            \item near: 900--3500 \AA{}
            \item far: 100--900 \AA{}
        \end{itemize}
        The 900 \AA{} break is because of the Lyman limit at 912 \AA{}.
        This is where neutral hydrogen is ionized, so the universe is largely
        opaque to wavelengths shorter than this.
    \item visual (V): 4000--7000 \AA{} (note that `V' is different from `optical',
        which is slightly broader: 3500--10000 \AA{}. The 3500 \AA{} cutoff
        is due to the Earth's atmosphere being opaque to wavelengths shorter
        than this).
    \item IR
        \begin{itemize}
            \item near: 1--5 $\mu$ (1--10 $\mu$ in online notes)
            \item mid: (10--100 $\mu$)
            \item far: 5--100 $\mu$ (100--1000 $\mu$)
        \end{itemize}
    \item sub-mm 500--1000 $\mu$
    \item microwave
    \item radio
\end{itemize}
Quantities of light:
\begin{itemize}
    \item Intensity (I) [erg $s^{-1}$ $\Omega^{-1}$ $\nu^{-1}$]
    \item Surface Brightness (SB)
    \item Flux (F)
    \item Luminosity (L)
\end{itemize}

\subsection*{Magnitudes and photometric systems}
\subsection*{Observed fluxes and the count equation}
\subsection*{Errors in photon rates}
\subsubsection*{Noise equation: how do we predict expected errors?}
\subsubsection*{Error propagation}
\subsubsection*{Determining sample parameters: averaging measurements}
\subsubsection*{Random errors vs systematic errors}

\end{document}
