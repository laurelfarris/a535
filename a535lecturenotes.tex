\documentclass[12pt]{article}
\usepackage{color}
\usepackage{hyperref}
\usepackage{amsmath}

\title{ASTR 535 Lecture Notes}
\author{Jon Holtzman}
\date{Spring 2016}

\addtolength{\oddsidemargin}{-0.875in}
\addtolength{\evensidemargin}{-0.875in}
\addtolength{\textwidth}{1.75in}
\addtolength{\topmargin}{-0.875in}
\addtolength{\textheight}{1.75in}


\begin{document}
\maketitle

\noindent course website: \textcolor{blue}{\url{http://astronomy.nmsu.edu/holtz/a535}}

\section*{Properties of light, magnitudes, errors, and error analysis}

\subsection*{Light}
Wavelength regimes:
\begin{itemize}
    \item gamma rays
    \item x-rays
    \item ultraviolent (UV)
        \begin{itemize}
            \item near: 900--3500 \AA{}
            \item far: 100--900 \AA{}
        \end{itemize}
        The 900 \AA{} break is because of the Lyman limit at 912 \AA{}.
        This is where neutral hydrogen is ionized, so the universe is largely
        opaque to wavelengths shorter than this.
    \item visual (V): 4000--7000 \AA{} (note that `V' is different from `optical',
        which is slightly broader: 3500--10000 \AA{}. The 3500 \AA{} cutoff
        is due to the Earth's atmosphere being opaque to wavelengths shorter
        than this).
    \item IR
        \begin{itemize}
            \item near: 1--5 $\mu$ (1--10 $\mu$ in online notes)
            \item mid: (10--100 $\mu$)
            \item far: 5--100 $\mu$ (100--1000 $\mu$)
        \end{itemize}
    \item sub-mm 500--1000 $\mu$
    \item microwave
    \item radio
\end{itemize}
Quantities of light:
\begin{itemize}
    \item Intensity (I) [erg $s^{-1}$ $\Omega^{-1}$ $\nu^{-1}$]
    \item Surface Brightness (SB): amount of energy $received$ in a unit surface
        element per unit time per unit frequency (or wavelength) from a unit
        solid angle in the direction ($\theta,\phi$), where $\theta$ is the angle
        away from the normal to the surface element, and $\phi$ is the azimuthal
        angle.
    \item Flux (F): amount of energy passing through a unit surface element
        in all directions, defined by
        \begin{equation}
            F_{\nu} = \int I_{\nu}\cos(\theta)\textrm{d}\Omega
        \end{equation}
        where d$\Omega$ is the solid angle element, and the integration is
        over the entire solid angle.
    \item Luminosity (L): $intrinsic$ energy emitted by the source per
        second ($\sim$ power). For an isotropically emitting source,
        \begin{equation}
            L = 4 \pi d^2 F
        \end{equation}
        where $d$ = distance to source.
\end{itemize}
What to measure for sources:
\begin{itemize}
    \item Resolved: directly measure surface brightness (intensity)
        distribution on the sky, usually over some bandpass or wavelength
        interval.
    \item Unresolved: measure the flux.
\end{itemize}
Questions:
\begin{itemize}
    \item What are the dimensions of the three quantities: luminosity,
        surface brightness (intensity), and flux?
    \item How do the three quantities depend on distance to the source?
    \item To what quantity is apparent magnitude of a star related?
    \item To what quantity is the absolute magnitude related?
\end{itemize}
Amount of light emitted is a function of wavelength, so we are often interested
in e.g.\ flux per unit wavelength or frequency, also known as $specific$
flux.
\begin{align*}
    \int F_{\nu} \textrm{d} \nu &= -\int F_{\lambda} \textrm{d} \lambda\\
    F_{\nu} &= -F_{\lambda} \textrm{d} \lambda\\
    &= F_{\lambda} \frac{c}{\nu^2}\\
    &= F_{\lambda} \frac{\lambda^2}{c}\\
\end{align*}
Note that a constant $F_{\lambda}$ implies a $non$-constant $F_{\nu}$
and vice versa.\\\\
\noindent Units: often cgs, magnitudes, Jansky (a flux density unit
corresponding to 10$^{-26}$ W m$^{-2}$ Hz$^{-1}$)\\\\
\noindent There are often variations in terminology\\\\
\noindent Terminology of measurements:
\begin{itemize}
    \item photometry (broad-band flux measurement)
    \item spectroscopy (relative measurement of fluxes at different wavelengths)
    \item Spectrophotomoetry (absolute measuremnet of fluxes at different wavelengths)
    \item astrometry (concerned with positions of observed flux)
    \item morphology (intensity as a function of position;
        often, absolute measurements are unimportant)
\end{itemize}
Generally, measure flux with photometry, and flux density with spectroscopy
(down to the resolution of the spectrograph). In pracetice, with most detectors,
measure photon flux (photon counting device), rather than every flux (bolometers).
The photon flux is given by
\begin{align*}
    photon flux &= flux photon^{-1} or\\
                &= energy photon^{-1}\\
                &= \int F_{\lambda} \frac{\lambda}{hc} \textrm{d} \lambda
\end{align*}

\subsection*{Magnitudes and photometric systems}
Magnitudes are related to flux (and SB and L) by
\begin{align*}
    m &= -2.5 \log \frac{F}{F_0}\\
      &= -2.5 \log F + 2.5 \log F_0
\end{align*}
where the coefficient of proportionality, $F_0$, depends on the definition
of photometric system; the quantity $-2.5 \log F_0$ may be referred to as
the photometric system zeropoint. Inverting, one gets:
\begin{equation*}
    F = F_0 \times 10^{-0.4\textrm{m}}
\end{equation*}
Just as fluxes can be represented in magnitude units, flux densities can be
specified by monochromatic magnitudes:
\begin{equation*}
    F_{\lambda} = F_0 (\lambda) \times 10^{-0.4 \textrm{m}(\lambda)}
\end{equation*}
although spectra are more often given in flux units than in magnitude units.
Note that it is possible that $F_0$ is a function of wavelength.\\

\noindent Since magnitudes are logarithmic, the $difference$ between
magnitudes corresponds to a ratio of fluxes; ratios of magnitudes are
generally unphysical. If one is just doing relative measurements of
brightness between objects, this can be done without knowledge of $F_0$
(or, equivalently, the system zeropoint); objects that differ in brightness
by $\Delta$M have the same ratio of brightness (10$^{-0.4 \Delta M}$)
regardless of what photometric system they are in. The photometric system
definitions and zeropoints are only needed when converting between calibrated
magnitudes and fluxes. Note that this means that if one references the
brightness of one object relative to that of another, a magnitude system
can be set up relative the brightness of the reference source. However, the
utility of a system when doing astrophysics generally requires an
understanding of the actual fluxes.

\newpage
\noindent \textcolor{magenta}{Monday, January 25}\\

\noindent There are three main types of magnitude systems in use in astronomy.
We start by describing the two simpler ones: the STMAG and the ABNU mag system.
In these simple systems, the reference flux is just a constant value in
$F_{\lambda}$ or $F_{\nu}$. However, these are not always the most widely used
systems in astronomy, because no natural source exists with a flat spectrum.\\

\noindent In the STMAG system, $F_{0,\lambda} = 3.60 \times 10^{-9}$
erg cm$^{-2}$ s$^{-1}$ \AA{}$^{-1}$, which is the flux of Vega at
5500 \AA{}; hence a star of Vega's brightness at 5500 \AA{} is defined to
have m=0. Alternatively, we can write
\begin{equation*}
    m_{STMAG} = -2.5 \log F_{\lambda} - 21.1
\end{equation*}
for $F_{\lambda}$ in cgs units.\\

\noindent In the ABNU system, things are defined for $F_{\nu}$ instead of
$F_{\lambda}$, and we have
\begin{equation*}
    F_{0,\nu} = 3.63
\end{equation*}



\subsection*{Observed fluxes and the count equation}
\subsection*{Errors in photon rates}
\subsubsection*{Noise equation: how do we predict expected errors?}
\subsubsection*{Error propagation}
\subsubsection*{Determining sample parameters: averaging measurements}
\subsubsection*{Random errors vs systematic errors}

\end{document}














