\documentclass[12pt]{article}
\usepackage[margin=1in]{geometry}
\usepackage{enumerate}

\setlength{\parindent}{0em}
\setlength{\parskip}{0.50em}

\begin{document}

\begin{enumerate}[1.]
  \item \textbf{Open the image SN17135\_r.0103.fits: gaia
  /home/apo/dec06/UT061215/SN17135\_r.0103.fits or ds9
  /home/apo/dec06/UT061215/SN17135\_r.0103.fits.}
  \begin{enumerate}[1.]
    \item We're looking at a region about 40$\times$40 square pixels.
      (This is an estimate obtained from moving the mouse around;
      I didn't see the exact size anywhere).
    \item The white line isn't part of the actual image?
    \item Done :)  
  \end{enumerate}
  
  \item \textbf{What do you see in the image?}
  \item \textbf{Look at the g and i band images of the same object
  (SN17135\_g.0101.fits and SN17135\_i.0105.fits). Compare and contrast
  with the r band image.}
  \item \textbf{Look through several of the other images in the same
  directory, making sure to include some of then SA... images, bias
  images, flat images, and focus images. Practice determining good
  display settings to use for all of the images. Notice as much detail
  as you can about the images, and ask if you don't know what
  something is.}
  \item \textbf{Do some of this again using another tool (gaia/ds9) tool,
  which has similar capability}
  \item \textbf{Try looking at some images taken with other instruments, e.g.:}



\end{enumerate}

\end{document}

