\documentclass[12pt]{article}
\usepackage[margin=1in]{geometry}
\usepackage{mdwlist}
\usepackage{hyperref}
\usepackage{color}
\setlength{\parindent}{0em}
\setlength{\parskip}{1em}
\pagenumbering{gobble}


\begin{document}
\begin{centering}
    {\large \textbf{Exposure Time Calculator: Weekly Report}}\\
    ASTR 535: Observational Astronomy\\
    25 March 2016\\
\end{centering}

\vspace{1cm}
Overview:
\vspace{-1.5em}
\begin{itemize*}
    \item Two forms of the exposure time calculator will be written: one in IDL and
        the other in Python.
        Caitlin and Agnar will be in charge of assembling the full form of the codes
        in Python and IDL, respectively.
        These will be modified as each group memeber submits their own piece
        of the code.
        A few of the extra-savvy members will attempt to break the code and diagnose
        problems, as laid out in the Google document.
\end{itemize*}

Accomplished:
\vspace{-1.5em}
\begin{itemize*}
    \item \href{http://astronomy.nmsu.edu/holtz/a535/project/}
        {\textcolor{blue}{Assignment specifications}} were reviewed.
    \item Individual
        functions and coding assignments were divided among group members.\\
        Details are available
        \href{https://docs.google.com/document/d/1BaF0ILo2b9gBW5ujPOmUyfkDtF6tfMoEgPA7_B5HZS8/edit}
        {\textcolor{blue}{here}}.
\end{itemize*}

Goals for next week:
\vspace{-1.5em}
\begin{itemize*}
    \item Each group member will submit a ``first draft'' of their assigned code.
\end{itemize*}


\end{document}
