\documentclass[12pt]{article}
\usepackage{enumerate}
\usepackage{color}
\usepackage{hyperref}

\title{ASTR 535 Lab notes}
\author{Jon Holtzman}
\date{Spring 2016}

\addtolength{\oddsidemargin}{-0.875in}
\addtolength{\evensidemargin}{-0.875in}
\addtolength{\textwidth}{1.75in}
\addtolength{\topmargin}{-0.875in}
\addtolength{\textheight}{1.75in}

\begin{document}
\maketitle

%\begin{enumerate}[I.]
\section*{Time, coordinate systems, observability tools}
\subsection*{Time Systems}
Systems of time: see \textcolor{blue}{\href{tycho.usno.navy.mil/systime.html}
        {Naval observatory reference}}
        for a full listing of different types of time.
    \subsubsection*{Solar Time}
        \begin{itemize}
            \item Time tied to position of Sun. Note the distinction
                between \textit{mean} solar time and \textit{apparent}
                solar time (the ``equation of time'' and the analemma).
            \item Most used solar time is Universal time.
                UT = local mean solar time at Greenwich = ``Zulu''.
                Tied to location of Sun, but average to ``mean sun''.
            \item Local time: accounts for longitude of observer.
                For practicality, legal time is split into time zones.
            \item In detail, official time is kept by atomic clocks
                (International Atomic Time, or TAI), and coordinated UT
                (UTC) is atomic time with leap seconds added to compensate
                for changes in earth's roation, where these are added to
                keep UTC wihin a second of solar time (UT1).
                See \textcolor{blue}
                %{\url{https://en.wikipedia.org/wiki/Universal_Time}}
                {\href{https://en.wikipedia.org/wiki/Universal_Time}
                {here}}
                for some details.
        \end{itemize}
    \subsubsection*{Sidereal time}
        \begin{itemize}
            \item Times based on position of stars, i.e. Earth's sidereal
                rotation period ~ 23h 56m 4s. Local sidereal time is GMST
                (Greenwich mean sidereal time) minus longitude. At the vernal
                equinox (time in sky when Sun crosses the celestial equator
                as its declination is increasing), sidereal time = UT.
                Difference between UT and GMST is one rotation (day) over
                the course of a year, so about 2 hours per month.
            \item Sideral is relevant for position of stars: stars come back
                to the same position every sidereal day.
        \end{itemize}
    \subsubsection*{Calendars}
        \begin{itemize}
            \item Standard calendar is Gregorian, with leap years, etc.
            \item For astronomy, it is simpler to keep track of days rather
                then year/month/day. Most dates given by the
                \url{Julian date} (number of days since UT noon, Monday,
                January 1, 4713 BC). Variations include modified Julian
                data (JD - 2400000.5 fewer digits and starts at midnight),
                heliocentric Julian date (JD adjusted to the frame of
                reference of the Sun, so can differ by up to 8.3 minutes).
            \item Note that repeating events are often described as an event
                \textit{ephemeris:} $t_i(event) = t_0 + i(period)$.
            \item The term \textit{ephemeris} is also used to describe how the
                position of an object changes over time, e.g. planetary
                ephemerides.
        \end{itemize}

\subsection*{Coordinate systems}
LPL website on \textcolor{blue}
{\href{http://spider.seds.org/spider/ScholarX/coords.html}
{astronomical coordinate systems}}
\subsubsection*{Celestial coordinate systems}
\textcolor{blue}
{\href{http://csep10.phys.utk.edu/astr161/lect/time/coordinates.html}{(diagram)}}
\begin{itemize}
    \item RA-DEC: tied to Earth rotation, longitude and latitude.
        Zero RA at vernal equinox
    \item ecliptic: tied to plane of Earth rotation around the Sun.
        Zero ecliptic longitude tied to vernal equinox.
    \item galactic: tied to plane of the Milky Way
\end{itemize}

\noindent At vernal equinox, RA = 12h crosses the meridian at midnight.\\

\noindent Note that for a celestial coordinate system tied to the Earth's rotation,
coordinates of an object change over time because of the changing direction
of the Earth's axis: precession and notation. Because of this, coordinates are
always specified for some reference equinox: J2000/FK5, B1950, etc.; if using
coordinates to point a telescope, you need to account for this (but generally,
telescope software does this on its own). Note distinction between equinox and
epoch, where the latter is relevant for objects that move (which everything does
at some level).\\

\noindent Transformations between systems straightforward from spherical
trigonometry.\\

\noindent Note the common usage of an Aitoff projection of the sky in celestial
coordinates, with location of ecliptic and galactic plane.

\subsubsection*{Local coordinate systems}
\begin{itemize}
    \item Equatorial: HA-dec. $HA=LST - \alpha. LST=GMST - longitude$.
        Note normal convention for HA is to get larger to the west, i.e.
        opposite of RA. Objects at zenith have $\delta=$ latitude of observer.
    \item Horizon: alt-az or zd-az
\end{itemize}
Local coordinates are important for pointing telescopes. Note that there are
various other effects that one has to consider for pointing a telescope at a
source of known celestial position: proper motion, precession, nutation,
``aberration of light'', parallax, atmosphereic refraction.

\subsection*{Finding positions of celestial objects}
\subsection*{Orientations of objects in the sky}
\subsection*{Observability}
\subsection*{Tools}
\subsection*{Exercises}

\newpage
\section*{Next Section}

%\end{enumerate}




\end{document}
