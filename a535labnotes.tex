\documentclass[12pt]{article}
\usepackage{enumerate}
\usepackage{color}
\usepackage{hyperref}

\title{ASTR 535 Lab notes}
\author{Jon Holtzman}
\date{Spring 2016}

\addtolength{\oddsidemargin}{-0.875in}
\addtolength{\evensidemargin}{-0.875in}
\addtolength{\textwidth}{1.75in}
\addtolength{\topmargin}{-0.875in}
\addtolength{\textheight}{1.75in}

\begin{document}
\maketitle

%\begin{enumerate}[I.]
\section*{Time, coordinate systems, observability tools}
\subsection*{Time Systems}
Systems of time: see \textcolor{blue}{\href{tycho.usno.navy.mil/systime.html}
        {Naval observatory reference}}
        for a full listing of different types of time.
    \subsubsection*{Solar Time}
        \begin{itemize}
            \item Time tied to position of Sun. Note the distinction
                between \textit{mean} solar time and \textit{apparent}
                solar time (the ``equation of time'' and the analemma).
            \item Most used solar time is Universal time.
                UT = local mean solar time at Greenwich = ``Zulu''.
                Tied to location of Sun, but average to ``mean sun''.
            \item Local time: accounts for longitude of observer.
                For practicality, legal time is split into time zones.
            \item In detail, official time is kept by atomic clocks
                (International Atomic Time, or TAI), and coordinated UT
                (UTC) is atomic time with leap seconds added to compensate
                for changes in earth's roation, where these are added to
                keep UTC wihin a second of solar time (UT1).
                See \textcolor{blue}
                %{\url{https://en.wikipedia.org/wiki/Universal_Time}}
                {\href{https://en.wikipedia.org/wiki/Universal_Time}
                {here}}
                for some details.
        \end{itemize}
    \subsubsection*{Sidereal time}
        \begin{itemize}
            \item Times based on position of stars, i.e. Earth's sidereal
                rotation period ~ 23h 56m 4s. Local sidereal time is GMST
                (Greenwich mean sidereal time) minus longitude. At the vernal
                equinox (time in sky when Sun crosses the celestial equator
                as its declination is increasing), sidereal time = UT.
                Difference between UT and GMST is one rotation (day) over
                the course of a year, so about 2 hours per month.
            \item Sideral is relevant for position of stars: stars come back
                to the same position every sidereal day.
        \end{itemize}
    \subsubsection*{Calendars}
        \begin{itemize}
            \item Standard calendar is Gregorian, with leap years, etc.
            \item For astronomy, it is simpler to keep track of days rather
                then year/month/day. Most dates given by the
                \url{Julian date} (number of days since UT noon, Monday,
                January 1, 4713 BC). Variations include modified Julian
                data (JD - 2400000.5 fewer digits and starts at midnight),
                heliocentric Julian date (JD adjusted to the frame of
                reference of the Sun, so can differ by up to 8.3 minutes).
            \item Note that repeating events are often described as an event
                \textit{ephemeris:} $t_i(event) = t_0 + i(period)$.
            \item The term \textit{ephemeris} is also used to describe how the
                position of an object changes over time, e.g. planetary
                ephemerides.
        \end{itemize}

\subsection*{Coordinate systems}
LPL website on \textcolor{blue}
{\href{http://spider.seds.org/spider/ScholarX/coords.html}
{astronomical coordinate systems}}
\subsubsection*{Celestial coordinate systems}
\textcolor{blue}
{\href{http://csep10.phys.utk.edu/astr161/lect/time/coordinates.html}{(diagram)}}
\begin{itemize}
    \item RA-DEC: tied to Earth rotation, longitude and latitude.
        Zero RA at vernal equinox
    \item ecliptic: tied to plane of Earth rotation around the Sun.
        Zero ecliptic longitude tied to vernal equinox.
    \item galactic: tied to plane of the Milky Way
\end{itemize}

\noindent At vernal equinox, RA = 12h crosses the meridian at midnight.\\

\noindent Note that for a celestial coordinate system tied to the Earth's rotation,
coordinates of an object change over time because of the changing direction
of the Earth's axis: precession and notation. Because of this, coordinates are
always specified for some reference equinox: J2000/FK5, B1950, etc.; if using
coordinates to point a telescope, you need to account for this (but generally,
telescope software does this on its own). Note distinction between equinox and
epoch, where the latter is relevant for objects that move (which everything does
at some level).\\ \noindent Transformations between systems straightforward from spherical
trigonometry.\\

\noindent Note the common usage of an Aitoff projection of the sky in celestial
coordinates, with location of ecliptic and galactic plane.

\subsubsection*{Local coordinate systems}
\begin{itemize}
    \item Equatorial: HA-dec. $HA=LST - \alpha. LST=GMST - longitude$.
        Note normal convention for HA is to get larger to the west, i.e.
        opposite of RA. Objects at zenith have $\delta=$ latitude of observer.
    \item Horizon: alt-az or zd-az
\end{itemize}
Local coordinates are important for pointing telescopes. Note that there are
various other effects that one has to consider for pointing a telescope at a
source of known celestial position: proper motion, precession, nutation,
``aberration of light'', parallax, atmosphereic refraction.

\subsection*{Finding positions of celestial objects}
\begin{itemize}
    \item \href{http://simbad.u-strasbg.fr/simbad/}{SIMBAD}:
        look up coordinates of many objects outside solar system by name, etc.,
        also provides much other reference information.
    \item \href{http://ned.ipac.caltech.edu}{NED}:
        NASA extragalactic database: galaxies, etc.
    \item solar system ephemerides: JPL
        \href{http://ssd.jpl.nasa.gov/horizons.cgi}{HORIZONS}
\end{itemize}

\subsection*{Orientations of objects in the sky}
Usually specified by position angle: angle of object in degrees from NS line,
measured counterclockwise.\\

\noindent An important observational position angle for spectroscopy:
\emph{parallactic angle}, the position angle of the line from zenith to
horizon.

\subsection*{Observability}
Obviously, to observe an object, one requires that it is visible above the
horizon. In general, one would liek to observe objects through the shortest
possible through Earth's atmosphere, i.e., when they are \emph{transiting}
(crossing the meridian, HA=0). The more atmosphere the light goes through,
the more losses due to atmospheric absorption/scattering (more severe at
shorter wavelengths), and the more image degradation from atmospheric seeing.
Of course, it doesn't make sense to wait for an object to transit if you
don't have anything else to do in the meantime; efficient use of telescope time
is the primary concern.
One \emph{airmass} is the amount of air directly above an observer.
If you are looking at the zenith, you are looking through one airmass.
Generally, most observers attempt to observe at airmasses
less than 2, i.e.\ within 60 degrees of zenith. Once you hit an airmass
of 3, the object is rapidly setting (except at very high declination).
Of course, for some solar system objects (objects near the sun), one has no
choice but to observe at high airmass.\\

\noindent Note that HA gives some indication of observability, but that
higher declination objects can be observed to higher HA than lower
declination objects. Roughly, at the celestial equator, an HA of 3 hours
is about an airmass of 2, and in many cases, one doesn't want to go much
lower in the sky.\\

\noindent Another issue with observability has to do with the Moon,
since it is harder to see fainter objects when the sky is brighter.
Moon brightness is related to its phase, and to a lesser extent, to
distance from your object. Of course, if the Moon is below the horizon,
it does not have an effect. So for planning observations of faint objects,
one also has to consider Moon phase and rise/set times. Note that the
sky brightness from the Moon is a function of wavelength, and at IR
wavelengths, it is not a very significant contributor to the total sky
brightness; so often, telescopes spend bright time working in the IR.

\subsection*{Tools}
Here are some useful software tools to do tasks related to coordinate systems
and observability, though there are others out there. Anything that
accomplishes the desired tasks adequately is fine to use; just make sure
you're not limited by the tools that you choose.
\begin{itemize}
    \item \href{}{\textcolor{blue}{skycalc/skycalendar}}:
        text based programs; skycalendar gives daily almanac, position of moon,
        etc.\ skycalc allows you to enter coordinates of an object and obtain
        observability information for any specified date. Other features included
        as well: coordinate transformation, position of planets.
    \item \href{}{\textcolor{blue}{JSkyCalc}}:
        (java-jar /home/local/java/JSkyCalc.jar): JAVA implementation
        of skycalc.
    \item \href{}{\textcolor{blue}{WCSTOOLS}}: full set of useful coordinate
        system programs, e.g.\ coordinate system transformation (command skycoor).
        Largely useful for use with coordinate system information in image
        headers (more later).
\end{itemize}

\subsection*{Exercises}
\begin{enumerate}[1.]
    \item Predict the RA at midnight for the first of every month.
        Try the command \texttt{skycalendar}
        (give yourself a wide terminal window first)
        to see how well you did.
    \item What time is it now? What is the sidereal time? What coordinates would
        it be most optimal to observe right now?
    \item When are the dark (no moon above horizon) first half nights in fourth
        quarter?
    \item APO schedules the 3.5m in half-night blocks (A and B), split at midnight
        (or 1am during daylight savings). What are the best half-nights in the next
        year (month and a half, e.g., Oct A, March B, etc.) to request to observe:
        \begin{itemize}
            \item Virgo cluster of galaxies (note central galaxy is M87,
                look up the coordinates).
            \item Galactic center (galactic coordinates are\ldots). Use command
                \texttt{skycoor} to convert galactic to equatorial
                (\texttt{skycoor} with no arguments gives syntax).
            \item Jupiter (look up its position using JPL HORIZONS).
        \end{itemize}
    \item Run \texttt{skycalc} (choose observatory A for APO.
        `?' gives list of command
        help. Look at r, d, y, and h commands). For the galactic center, what is
        the maximum amount of time it can be observed at an airmass of less than
        2.5? How about the Virgo cluster? Why are these different?
    \item Run \texttt{jskycalc}. Play with all of the buttons! What planets will be
        visible fall 2011, and at what times of night? Note that you can load
        files with a list of coordinates, and you can make airmass observability
        charts for them.
    \item Make a plan for a 3 half-night observing run during early/mid November,
        A halves, when we are likely to take our APO trip. The plan should
        include a list of objects for each night with a tentative order of
        observation, taking into account how much times needs to be spent on
        each object. Our projects are still TBD, but will likely include:
        \begin{itemize}
            \item ARCES (echelle spectrograph) observations of several stars
                in the Kepler field (look this up if you don't know what it is,
                or where it is) to monitor velocity of eclipsing binaries
                (Patrick) and to measure rotation of rapid rotators (Dmitry).
            \item ARCES observations of bright stars observed by the SDSS-III
                APOGEE survey; these are located all over the sky.
            \item DIS observations of a candidate high velocity star from SEGUE
                (Young Sun) at RA=3h54m DEC=-6d14m.
            \item SPICAM observation of candidate low mass dwarf galaxies from
                SDSS (equatorial candidates between 21h and 3h RA).
            \item If it is clear, we will likely have time for some additional
                projects. Can you come up with some?
        \end{itemize}
    \item Prepare a web page with the plan, including relevant information:
        coordinates of objects, finder images if necessary, links to tabulated
        spectra, instrument manuals, etc.
\end{enumerate}

\newpage
\section*{Next Section}

%\end{enumerate}















\end{document}
