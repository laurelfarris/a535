\documentclass[12pt]{article}
\usepackage[margin=1in]{geometry}
\usepackage{fancyhdr}
\pagestyle{fancy}
\usepackage{enumerate}

\setlength{\parindent}{0em}
\setlength{\parskip}{0.50em}

\lhead{Laurel Farris}
\chead{ASTR 535 - Lab 03}
\rhead{12 February 2016}

\begin{document}

\begin{enumerate}[1.]
    \item \textbf{Read in several images: SN17135\_r.0103.fits,
    flat\_r.0015.fits, bias.0001.fits. Display them from inside a
    software package (IDL/atv or Python/pyds9/tv/imshow) and get
    familiar with using that.} 
    \item \textbf{Construct image histograms for several images:
    SN17135\_r.0103.fits, flat\_r.0015.fits, bias.0001.fits. See if you
    can explain where all of the different values are coming from.}
    \item \textbf{Choose some sky subregions in SN17135\_r.0103.fits,
    excluding objects in the field. Look at some pixel histograms of
    these regions. Determine the mean and standard deviation in the
    regions. What do you expect for the standard deviations? Are the
    standard deviations what you expect? Is the standard deviation in
    a region with an object a meaningful/useful quantity to look at?}
    \item \textbf{Note the overscan region at the right side of the
    image (look at SN17135\_r.0103.fits or flat\_r.0015.fits). This
    gives the bias level, a constant that is added to the image to
    insure that readout noise does not lead to any negative values.
    Determine the mean in the bias region, as well as the standard
    deviation, using image statistics. What do you think the standard
    deviation is giving you information about?}
    \item \textbf{Using the mean bias level, use image arithmetic to
    subtract this level off of the entire image. How will this affect
    the previously calculated means and standard deviations in the
    image subregions? Are the standard deviations what you expect?}
    \item \textbf{For many astronomical detectors, the numbers
    recorded are not the number of photons incident on each pixel, but
    are related to the number of photons by a multiplicative factor,
    called the gain. For SPICAM, the gain is about 4, i.e., the number
    of photons incident on each pixel is given by 4 times the data
    numbers. Convert the images to photon counts by multiplying the
    image by this factor. How will this affect the previously
    calculated means and standard deviations?}
    \item \textbf{Now that you've subtracted bias level and accounted
    for gain, are the standard deviations in your subregions closer to
    your expectations?}
    \item \textbf{There is still some extra scatter coming from
    pixel-to-pixel sensitivity variations. Looking at
    flat\_r.0015.fits, estimate the level of this variation.}
    \item \textbf{To correct for pixel-to-pixel variation, we divide
    raw image by a normalized flat field. Normalize the flat field by
    dividing it by the mean in the central region, then divide the
    object frame by the normalized flat field. Recalculate the image
    statistics in the subregion and comment on how they have changed.}
    \item \textbf{Write a program to loop over ALL images taken on
    UT061215 and compute and output the mean bias level and the mean
    sky level ( for Python, check mmm routines (in tv.tv); for IDL,
    check out Astronomy Users Library sky command). Make a single plot
    of vertical cross sections in the overscan regions, averaging over
    the width of the overscan region.}

\end{enumerate}

\end{document}

