\documentclass{article}
\usepackage[margin=1in]{geometry}
\usepackage{enumerate}
\usepackage{color}
\usepackage{mdwlist}
\usepackage{hyperref}

\setlength{\parindent}{0em}
\setlength{\parskip}{0.5em}

\title{\vspace{-0.75in}ASTR 535 Lab notes}
\author{Jon Holtzman}
\date{Spring 2016}

\begin{document}
\maketitle

\section*{Time, coordinate systems, observability tools}
\subsection*{Time Systems}
Systems of time: see \textcolor{blue}{\href{tycho.usno.navy.mil/systime.html}
        {Naval observatory reference}}
        for a full listing of different types of time.
\subsubsection*{Solar Time}
Time tied to position of Sun; based on amount of time it
takes for the sun to return to the same position in the sky
(aka days).
Note the distinction
between \emph{mean} solar time (clock time)
and \emph{apparent}
solar time (sundial, the ``equation of time'' and the analemma).

Most used solar time is Universal time.
UT = local mean solar time at Greenwich = ``Zulu''.
Tied to location of Sun, but average to ``mean sun''.

Local time: accounts for longitude of observer.
For practicality, legal time is split into time zones.

In detail, official time is kept by atomic clocks
(International Atomic Time, or TAI), and coordinated UT
(UTC) is atomic time with leap seconds added to compensate
for changes in earth's roation, where these are added to
keep UTC within a second of solar time (UT1).
See \textcolor{blue}
%{\url{https://en.wikipedia.org/wiki/Universal_Time}}
{\href{https://en.wikipedia.org/wiki/Universal_Time}
{here}} for some details.

\subsubsection*{Sidereal time}
Times based on position of stars, i.e. Earth's sidereal
rotation period $\sim$ 23h 56m 4s. Local sidereal time is GMST
(Greenwich mean sidereal time) minus longitude. At the vernal
equinox (time in sky when Sun crosses the celestial equator
as its declination is increasing), sidereal time = UT.
Difference between UT and GMST is one rotation (day) over
the course of a year, so about 2 hours per month.

Sideral is relevant for position of stars: stars come back
to the same position every sidereal day. As we'll see
below, \textcolor{red}{a given star crosses the meridian when the
local sidereal time equals the right ascension of the
star.}

\subsubsection*{Calendars}

Standard calendar is Gregorian, with leap years, etc.

For astronomy, it is simpler to keep track of days rather
    then year/month/day. Most dates given by the
    \href{https://en.wikipedia.org/wiki/Julian_day}
    {\textcolor{blue}{Julian date}}
    (number of days since UT noon, Monday,
    January 1, 4713 BC). Variations include modified Julian
    data (JD - 2400000.5 fewer digits and starts at midnight),
    heliocentric Julian date (JD adjusted to the frame of
    reference of the Sun, so can differ by up to 8.3 minutes).
    Heliocentric JD is the amount of time it would take a
    pulse of light to arrive at the sun.

Note that repeating events are often described as an event
    \emph{ephemeris:} $t_i(event) = t_0 + i(period)$.

The term \emph{ephemeris} is also used to describe how the
    position of an object changes over time, e.g. planetary
    ephemerides.

\subsection*{Coordinate systems}
LPL website on \textcolor{blue}
{\href{http://spider.seds.org/spider/ScholarX/coords.html}
{astronomical coordinate systems}}
\subsubsection*{Celestial coordinate systems}
\textcolor{blue}
{\href{http://csep10.phys.utk.edu/astr161/lect/time/coordinates.html}{(diagram)}}
\begin{itemize}
    \item RA-DEC: tied to Earth rotation, longitude and latitude.
        Zero RA at vernal equinox
    \item ecliptic: tied to plane of Earth rotation around the Sun.
        Zero ecliptic longitude tied to vernal equinox.
    \item galactic: tied to plane of the Milky Way
\end{itemize}

At vernal equinox, RA = 12h crosses the meridian at midnight.

Note that for a celestial coordinate system tied to the
Earth's rotation,
coordinates of an object change over time because of the changing direction
of the Earth's axis: precession and nutation. Because of this, coordinates are
always specified for some reference equinox: J2000/FK5, B1950, etc.; if using
coordinates to point a telescope, you need to account for this (but generally,
telescope software does this on its own). Note distinction between equinox and
epoch, where the latter is relevant for objects that move (which everything does
at some level).

Transformations between systems straightforward from spherical
trigonometry.

Note the common usage of an Aitoff projection (equal areas)
of the sky in celestial
coordinates, with location of ecliptic and galactic plane.
Software tools (Python, projection=``aitoff'' in subplot,
IDL: aitoff and aitoff\_grid in Astronomy users library).

\subsubsection*{Local coordinate systems}
\begin{itemize}
    \item Equatorial: HA-dec. $HA=LST - \alpha.\ LST=GMST - longitude$.
        Note normal convention for HA is to get larger to the west, i.e.
        opposite of RA. Objects at zenith have $\delta=$ latitude of observer.
    \item Horizon: alt-az or zd-az
\end{itemize}
Local coordinates are important for pointing telescopes. Note that there are
various other effects that one has to consider for pointing a telescope at a
source of known celestial position: proper motion, precession, nutation,
``aberration of light'', parallax, atmosphereic refraction.

\subsection*{Finding positions of celestial objects}
\begin{itemize}
    \item \href{http://simbad.u-strasbg.fr/simbad/}
        {\textcolor{blue}{SIMBAD}}:
        look up coordinates of many objects outside solar system
        by name, etc., also provides much other reference information.
    \item \href{http://vizier.u-strasbg.fr/viz-bin/VizieR}
        {\textcolor{blue}{VizieR catalog database}}
        Database of astronomical catalogs, with search and download
        possibilities.
    \item \href{http://ned.ipac.caltech.edu}
        {\textcolor{blue}{NED}}:
        NASA extragalactic database: galaxies, etc.
    \item solar system ephemerides: JPL
        \href{http://ssd.jpl.nasa.gov/horizons.cgi}
        {\textcolor{blue}{HORIZONS}}
\end{itemize}

\subsection*{Orientations of objects in the sky}
Usually specified by position angle: angle of object in degrees from NS line,
measured counterclockwise.

An important observational position angle for spectroscopy:
\emph{parallactic angle}, the position angle of the line from zenith to
horizon.

\subsection*{Observability}
In general, one would like to observe objects through the shortest
possible path through Earth's atmosphere, i.e., when they are \emph{transiting}
(crossing the meridian, HA=0). The more atmosphere the light goes through,
the more losses due to atmospheric absorption/scattering (more severe at
shorter wavelengths), and the more image degradation from atmospheric seeing.
Of course, it doesn't make sense to wait for an object to transit if you
don't have anything else to do in the meantime; efficient use of telescope time
is the primary concern.
One \emph{airmass} is the amount of air directly above an observer.
If you are looking at the zenith, you are looking through one airmass.
Generally, most observers attempt to observe at airmasses
less than 2, i.e.\ within 60 degrees of zenith. Once you hit an airmass
of 3, the object is rapidly setting (except at very high declination).
Of course, for some solar system objects (objects near the sun), one has no
choice but to observe at high airmass.

Note that HA gives some indication of observability, but that
higher declination objects can be observed to higher HA than lower
declination objects. Roughly, at the celestial equator, an HA of 3 hours
is about an airmass of 2, and in many cases, one doesn't want to go much
lower in the sky.

Another issue with observability has to do with the Moon,
since it is harder to see fainter objects when the sky is brighter.
Moon brightness is related to its phase, and to a lesser extent, to
distance from your object. Of course, if the Moon is below the horizon,
it does not have an effect. So for planning observations of faint objects,
one also has to consider Moon phase and rise/set times. Note that the
sky brightness from the Moon is a function of wavelength, and at IR
wavelengths, it is not a very significant contributor to the total sky
brightness; so often, telescopes spend bright time working in the IR.

\subsection*{Tools}
Here are some useful software tools to do tasks related to coordinate systems
and observability, though there are others out there. Anything that
accomplishes the desired tasks adequately is fine to use; just make sure
you're not limited by the tools that you choose. These are available
on the Astronomy Linux cluster; you can probably install them on your
laptop, but they will probably not be there by default.
\begin{itemize}
    \item \href{http://physics.dartmouth.edu/}
        {\textcolor{blue}{skycalc/skycalendar}}:
        text based programs, installed on our Linux cluster (link is
        to source code if you wish to install on your laptop).
        skycalendar gives daily almanac, position of moon, etc.
        skycalc allows you to enter coordinates of an object and
        obtain observability information for any specified date. Other
        features included as well: coordinate transformation, position
        of planets.
    \item \href{http://physics.dartmouth.edu/}
        {\textcolor{blue}{JSkyCalc}}:
        (java-jar /home/local/java/JSkyCalc.jar): JAVA implementation
        of skycalc, also installed on the Astronomy cluster (and
        available for download).
    \item \href{http://tdc-www.harvard.edu/wcstools/}
        {\textcolor{blue}{WCSTOOLS}}: full set of useful coordinate
        system programs, e.g.\ coordinate system transformation
        (command skycoor). Largely useful for use with coordinate
        system information in image
        headers (more later). Installed on the astronomy cluster.
    \item Python: \href{http://docs.astropy.org/en/stable/coordinates/}
        {\textcolor{blue}{astropy.coordinates}},
        IDL:
        \href{http://idlastro.gsfc.nasa.gov/ftp/pro/astro/euler.pro}
        {\textcolor{blue}{euler}} in Astronomy users library.
\end{itemize}


\subsection*{Exercises}
\begin{enumerate}[1.]
    \item \textbf{Predict the RA crossing the meridian at midnight for the
    first of every month. Try the command skycalendar (on the cluster,
    unless you download it yourself for your laptop) - give yourself a
    wide terminal window first - to see how well you did}

    \item \textbf{What time is it now? What is the sidereal time? What
    coordinates would it be most optimal to observe right now?}

    \item \textbf{When are the dark (no moon above horizon) first half
    nights in first quarter ? }

    \item \textbf{APO schedules the 3.5m in half-night blocks (A and
    B), split at midnight (or 1am during daylight savings). What are
    the best half-nights in the next year (month and half, e.g., Oct
    A, March B, etc.) to request to observe:}
        \begin{itemize}
            \item \textbf{Virgo cluster of galaxies (note central
            galaxy is M87, look up the coordinates)}
            \item \textbf{Galactic center (galactic coordinates are
            .... ask if you don't know!). You can use command skycoor
            (or Python or IDL tools) to convert galactic to equatorial
            (skycoor with no arguments gives syntax). }
            \item \textbf{Jupiter (look up its position using JPL
            HORIZONS) }
        \end{itemize}

    \item \textbf{Run skycalc (choose observatory A for APO, ? gives
    list of command help, look at r, d, y, and h commands). For the
    galactic center, what is the maximum amount of time it can be
    observed at an airmass of less than 2.5? How about the Virgo
    cluster? Why are these different?}

    \item \textbf{Run jskycalc. Play with all of the buttons! What
    planets will be visible spring 2016, and at what times of night?
    Note that you can load files with a list of coordinates, and you
    can make airmass observability charts for them.}

    \item \textbf{Start to outline plan for an 3 half-night observing
    run during late March A halves, when we are taking our APO trip.
    Eventually, the plan should include a list of objects for each
    night with a tentative order of observation, taking into account
    how much time needs to be spent on each object. Our projects are
    still TBD, but will likely include observations with multiple
    instruments.}
        \begin{itemize}
            \item \textbf{Determine the approximate range of RAs that
            we will be able to observe. }
            \item \textbf{Given the NMSU 1st quarter proposals, which
            of them might we be able to make some observations for?}
            \item \textbf{If you have other ideas for projects, start
            to tabulate them. (Sten/Diane stars for APOGEE
            calibration/neutron capture calibration, Triplespec RR
            Lyrae RV curves Drew Be stars)}
            \item \textbf{Start to prepare a joint web page with the
            plan, including relevant information: coordinates of
            objects, finder images if necessary, links to tabulated
            spectra, instrument manuals, etc. etc. }
        \end{itemize}

    \item \textbf{Look up the catalog Globular Clusters in the Milky
    Way in VizieR and download it (make sure to get all of the rows).}
        \begin{itemize}
            \item \textbf{Plot the locations in an Aitoff projection
            of equatorial coordinates. Can you detect Galactic
            structure?}
            \item \textbf{What clusters would be possible to observe
            during our March run? }
            \item \textbf{Convert coordinates to galactic coordinates
            and plot in an Aitoff projection.}
        \end{itemize}

\end{enumerate}

\newpage
\textcolor{magenta}{Friday, February 5}

\section*{Image display and graphical file-based display tools}
\subsection*{Image Display}
\subsection*{Quick introduction to astronomical image file format}
\subsection*{Standalone display tools}
\subsection*{Basic display operation}
\subsection*{Exercises}


\textcolor{magenta}{Friday, February 26}

\section*{Astronomical image processing packages: Introduction and
IRAF basics}

\subsection*{IRAF/DS9 basic operation}
\begin{itemize*}
    \item One time only: run \texttt{mkiraf}, which creates \texttt{login.cl}.
        This file can be customized at a later time if you have settings you want
        to start with every time.
        \begin{itemize*}
            \item to enable a larger frame buffer for display, uncomment and
                modify line: \texttt{stdimage = imt2048}
        \end{itemize*}
    \item The preferred method of running IRAF in the modern era is using the
        PYTHON interface, \texttt{pyraf}:
        \begin{itemize*}
            \item You can use pyraf via a normal python interface:

                \texttt{from pyraf import iraf}

                You will then need to use standard PYTHON styntax, rather
                than the old IRAF \texttt{cl} syntax.
            \item \texttt{pyraf} actually runs a front-end interpreter to
                emulate the original IRAF command-line interface. This is
                convenient for previous users and for some tasks, but
                ``hides'' the Python interpreter and its power.
        \end{itemize*}
    \item For displaying images, start an image display tool, i.e.\ DS9,
        in the background: \texttt{ds9 \&}. Be aware that the stdimage that is set
        in the login.cl file may limit the maximum size of the image that will
        be displayed.
    \item help:
        \begin{itemize*}
            \item there is an internal \texttt{.help} command.
            \item IRAF help
            \item tutorials
        \end{itemize*}
    \item basics:
        \begin{itemize*}
            \item IRAF contains many programs for astronomical analysis. These
                are grouped into \emph{packages}, and individual commands are
        \end{itemize*}
\end{itemize*}


\section{Astronomical image processing: Introduction and basics}

Various software packages have been developed for astronomical image processing,
e.g.\
\begin{itemize*}
    \item IRAF (links and such)
\end{itemize*}
Pros and cons: availability, cost, GUI/command line, data handling
(disk vs. memory), speed, ease of use (e.g., keywords vs. parm files),
language and access to existing code, ability to add new code,
scripts/ procedures (internal control language).

Image processing package as a tool: tools can be incredibly useful,
but sometimes significant investment in understanding/learning your
tool really increases its utility. But also, in the long run, it's a
tool, and you shouldn't be limited in what you choose to do by the
tool you are comfortable with, so always keep open the possibility of
other tools, or improving the capability of a tool!

What should you learn? These days, many instruments require rather
involved tasks for reducing data. Often, the instrument team or
observatory supplies routines (in some package) for doing these tasks.
Generally, it is may be easier to use these routines rather than
reprogram them using your favorite tool. So you are probably in the
position of having to be comfortable with multiple tools, but you
should also probably take the time to become an expert in at least
one.

An alternative way to look at things is that to be at the forefront,
you will likely be working
with new instruments and/or new techniques. Using standard analysis
may be unlikely to take the most advantage, or even work at all, with
new data. So you want to be in the position of having the flexibility
to develop tools yourself.

There are several programming environments that make it fairly simple
to work with astronomical data. Here, we'll provide an introduction to
two of the more popular environments in the US: Python (especially
useful in conjuction with PyRAF) and IDL. Working in one of these
environments allows you to script the use of existing routines, and
also to develop your own routines. Also extremely important to have
tools to be able to explore data.

\subsection{Getting started with Python}
\subsubsection{Basics}
\begin{itemize*}
    \item Start python using \texttt{ipython -matplotlib}
    \item Python works with \emph{objects}. All objects can have
        attributes and methods.
    \item Get information:
        \begin{itemize*}
            \item \texttt{type(var)} gives type of variable
            \item \texttt{var?} gives information on variable (iPython
                only)
            \item \texttt{var.<tab>} gives information on variable
                attributes and methods.
        \end{itemize*}
    \item Python as a language
        \begin{itemize*}
            \item conditionals via if/elif/else
            \item looping via for, while
        \end{itemize*}
\end{itemize*}

\subsubsection{File I/O with astropy}
\begin{itemize*}
    \item FITS: header/data, data types, HDUList, etc.
        \begin{itemize*}
            \item \texttt{from astropy.io import fits}
            \item \texttt{hd=fits.open(\emph{filename})}
                returns HDULIST
            \item \texttt{hd[0].data} is data from initial HDU
            \item \texttt{hd[0].header} is header from initial HDU
        \end{itemize*}
    \item ASCII:
        \begin{itemize*}
            \item \texttt{from astropy.io import ascii}
            \item \texttt{a=ascii.read(\emph{filename})} returns Table with columns.
        \end{itemize*}
\end{itemize*}

\subsubsection{Image statistics}
\begin{itemize*}
    \item numpy array methods; e.g.:
        \begin{itemize*}
            \item \texttt{data.sum()}: total
            \item \texttt{data.mean()}: mean
            \item \texttt{data.std()}: standard deviation
        \end{itemize*}
    \item subsections: data[y1:y2,x1:x2]
\end{itemize*}

\subsubsection{Image display}
\begin{itemize*}
    \item primitive display via imshow
        \begin{itemize*}
            \item plt.imshow(hd[0].data,vmin=min,vmax=max)
        \end{itemize*}
    \item display using pyds9,
        \begin{itemize*}
            \item from pyds9 import *
            \item d=DS9() (opens a DS9 window, associates with object d)
            \item d.set("fits filename") (display from file)
            \item d.set\_pyfits(hd) (display from HDULIST)
            \item d.set\_np2arr(hd[0].data) (display from numpy array)
            \item d.set("scale limits 400 500") (sets display range)
            \item command list
        \end{itemize*}
    \item display with tv
        \begin{itemize*}
            \item import os
            \item os.environ["PYTHONPATH"] = /home/holtz/python
            \item from tv.tv import *
            \item t=TV()
            \item t.tv(hd[0],min=400,max=500)
            \item t.tv(hd[0].data)
            \item zoom, pan, colorbar
            \item blinking image buffers with +/-
        \end{itemize*}
\end{itemize*}

\subsubsection*{Plotting}
\begin{itemize*}
    \item \texttt{plt.figure}
    \item \texttt{plt.plot(hd[0].data[:,100])} for a plot along column 100
    \item \texttt{plt.plot(hd[0].data[500,:])} for a plot along row 500
\end{itemize*}

\subsubsection*{Histogram}
\texttt{plt.hist(data.flatten(),[bins=n],[bins=np.arange(min,max,delta)],
[log=True])}

%----------------------------------------------------------------------------%
\section*{Introduction to CCD images and basic CCD data reduction}
%----------------------------------------------------------------------------%
\section*{Astronomical image processing packages: Introduction and IRAF basics}
%----------------------------------------------------------------------------%
\section*{Astronomical image processing/reduction: Basic tools}
\textcolor{magenta}{Friday, April 1, 2016}

When observing, a bare minimum requirement is the ability to look at
your data. In many cases, however, it is preferable to have tools to
do some quick image manipulation and analysis, and these will be
required for image reduction/analysis. It's best if these are easily
available so that you are likely to encounter them in most computing
situations, and ideally, could access them on your laptop if you have
one.

In the current computing climate, I would recommend using Python tools
wherever possible. For some analysis, IRAF routines provide a lot of
developed routines, so if IRAF installed, these can be useful; I would
recommend using them from a Python environment to be able to take
advantage of native Python features.

For image display, \texttt{ds9} is probably the best choice, although there may
be alternatives.

Our goal is to work towards reduction of all of our APO images.

\subsection*{Getting started}
\begin{itemize}
\item Start ds9 in the background
\begin{verbatim}
ds9 &
\end{verbatim}
\item Start an iPython session
\begin{verbatim}
ipython --matplotlib
\end{verbatim}

\item Import standard Python packages
\begin{verbatim}
import numpy as np
import matplotlib.pyplot as plt
import pyds9
\end{verbatim}
(note that you can put these in a
\texttt{/.ipython/profile\_default/startup/00startup.py}
script to load every time you start ipython.)

\item Import useful astropy routines
\begin{verbatim}
from astropy.io import fits
\end{verbatim}

\item Create \texttt{login.cl} file
If IRAF is available, make sure you have a login.cl file. If you don't:
\begin{verbatim}
mkiraf # note this is a UNIX command, not a python command
\end{verbatim}
and edit the \texttt{login.cl} file to set \texttt{stdimage=imt2048},
or copy a \texttt{login.cl} file from a previous directory.
\begin{verbatim}
from pyraf import iraf
\end{verbatim}
in which case, you will need to call iraf routines using
\texttt{iraf.routine\_name()}
which makes it clear that they are IRAF routines. If you want to
enter the routine names without the \texttt{iraf.} prefix, type
\begin{verbatim}
from pyraf.iraf import *
\end{verbatim}
\end{itemize}

\subsection*{Reading images}
\begin{itemize}
\item Read image into variable:
\begin{verbatim}
im=fits.open(filename)[0]
\end{verbatim}
Note that this reads the first extension ([0]) into a HDU object, with
im.header containing the header, and im.data containing the data
\item For convenience, you might want to:
\begin{itemize}
\item Set up a variable with the directory name for the images,
to avoid having to retype it:
\begin{verbatim}
imdir='/pathtoimage directory/'
im=fits.open(imdir+'nameoffile')[0]
\end{verbatim}
\item Set up a symbolic link to the directory with the images,
to avoid having to retype it:
\begin{verbatim}
%ln -s /pathtoimage directory/ raw    # UNIX command
im=fits.open('raw/nameoffile')[0]
\end{verbatim}
\end{itemize}
\item IDL: im=mrdfits('filename')
\end{itemize}


\subsection*{Displaying images}
\begin{itemize}
\item Direct from memory (variable):
\begin{verbatim}
d=DS9()   # to open display
hd=fits.open(filename)   # puts HDUlist of file into hd
d.set_pyfits(hd)  #  display from HDUList variable
d.set_np2arr(hd[0].data)  # display from numpy array
d.set("scale limits 400 500")  (sets display range)
\end{verbatim}
You might want to write yourself a simple Python function to display
and scale with a single simple command!
\item Direct from disk, using IRAF display:
\begin{verbatim}
iraf.display(imdir+'nameoffile')
\end{verbatim}
If you wish to control display parameters (recommended):
\begin{verbatim}
iraf.display(imdir+'nameoffile',zrange='No',scale='No',z1=low,z2=high)
\end{verbatim}
where low, high are the values you want for color mapping. If you want
to have your values set by default, you can:
\begin{verbatim}
iraf.epar('display')
\end{verbatim}
and set zrange and scale to 'No', or alternatively:
\begin{verbatim}
iraf.display.setParam('zrange=no')
iraf.display.setParam('zscale=no')
\end{verbatim}
\item IDL: atv,im,[min=min,max=max]
\end{itemize}


\subsection*{Image inspection}
\begin{itemize}
\item image cross sections:
    \begin{itemize}
        \item Python:
            \begin{verbatim}
plt.plot(im.data[:,500])  # plots row 500
plt.plot(im.data[500,:)  # plots column 500
            \end{verbatim}
        \item IRAF: \emph{implot} task.
            \begin{itemize}
                \item See plot window commands ('?')
                \item `l' and `c' for line (row) and column plots,
                    as determined by cursor location
            \end{itemize}
        \item IDL: plot,im[*,500]
    \end{itemize}

\item Image histogram:
    \begin{itemize}
        \item Python:
            \begin{verbatim}
plt.hist(im.data.flatten(),bins=....)
            \end{verbatim}
        \item IRAF imhist. Look at the parameter file for options. Note that you
            can specify image subsections using
            \texttt{filename[x1:x2,y1:y2]}
        \item IDL: plothist,im
    \end{itemize}

\item Image statistics:
    \begin{itemize}
        \item Python: use numpy array methods: mean, sum and std, e.g.,
            \begin{verbatim}
mean=im.data[400:600,400:600].mean()
tot=im.data[400:600,400:600].sum()
sig=im.data[400:600,400:600].std()
            \end{verbatim}
        \item IRAF imstat. Look at the parameter file. Note you can specify
            image subsections as above.
        \item IDL: MEAN(), STDEV() functions
\end{itemize}

\item Image arithmetic:
    \begin{itemize}
        \item Python: just use normal arithmetic, e.g.:
            \begin{verbatim}
a=im1.data-bias
b=im1.data=im2.data
            \end{verbatim}
        \item IRAF imarith: file based. You can do arithmetic with images and
            constants, or with multiple images. For example: imarith
            file1.fits - 363 will subtract a constant of 363 from the image,
            imarith file1.fits / file2.fits will divide file1 by file2 (on a
            pixel-by-pixel basis).
        \item IDL: normal array arithmetic
    \end{itemize}

\item Interactive inspection of stellar images:
    \begin{itemize}
        \item Python: someone needs to write some tools!
        \item IRAF imexam: need to display image with iraf.display() first. Note
            `a', `r', and `m' keys, `?' for help (note you have to exit help
            to get interactive cursor!), `q' for quit.
        \item IDL: atv
    \end{itemize}
\end{itemize}

\section*{Data reduction}
\textcolor{magenta}{Friday, April 15, 2016}

Our goal is to understand all of the steps and issues involved with
data reduction and how they may be dealt with when people reduce data,
and to try to avoid, as much as possible, ``black-box" recipes for
reducing data.

To be able to capture the process, it is best if data reduction
efforts always be scripted, so that you have a record of what you did,
and a resource to look back on the next time you have to do it again!

Your goal is to deliver basic data reduction scripts for the standard
stars observed with ARCTIC and DIS

\subsection*{Basic data reduction}
\begin{itemize*}
    \item Overscan subtraction
        \begin{itemize*}
            \item Determine overscan region location
            \item Determine whether constant overscan (subtraction of a single value) is
                appropriate, or if not, consider possibilities:
                \begin{itemize*}
                    \item Fit to overscan as a function of row
                    \item Median overscan as a function of raw
                \end{itemize*}
            \item Remove overscan
                \begin{itemize*}
                    \item Using image arithmetic
                    \item Using IRAF: ccdproc (note overscan options)
                \end{itemize*}
        \end{itemize*}
    \item Superbias (zero) frame construction
        \begin{itemize*}
            \item Inspect overscan-subtracted bias frames. If there is repeatable
                structure in these, construct a superbias frame by combining
                overscan-subtracted bias frames:
                \begin{itemize*}
                    \item Using image arithmetic
                    \item Using IRAF: zerocombine
                    \item Note that there are multiple options for combining stacks of frames,
                        to avoid contamination by outliers, resulting biases, noise
                        minimization, etc: mean, median, max-reject, min-max reject, sigma
                        clipping, etc. Median is a simple algorithm that is fairly robust if
                        not perfectly optimal.
                \end{itemize*}
            \item Note that any noise in your superbias frame will be propagated to
                every image you reduce, hence the desire to combine many individual
                bias frames, and only to use a superbias if there is repeatable
                structure to subtract!
        \end{itemize*}
    \item Flat field construction
        \begin{itemize*}
            \item You will need to construct separate flat fields for each
                filter/configuration that you use
            \item Flat fields should be normalized before combining to account for
                variations in lamp/sky brightness
            \item Final flat fields should be normalized such that dividing by them does
                not change the overall mean level significantly, so that noise can
                still be calculated using the observed number of
                counts. Don't want to change numbers much because want
                to measure uncertainty on brightness later
            \item Making flats:
                \begin{itemize*}
                    \item Using image arithmetic
                    \item Using IRAF: flatcombine
                    \item Again, there are many frame combination options.
                \end{itemize*}
        \end{itemize*}
\end{itemize*}

\subsection*{Basic spectroscopic calibration}
\begin{enumerate*}
    \item normal CCD processing: overscan, (bias, dark). (Note that
        Triplespec is not a CCD, so requires normal IR detector
        processing: dark/bias subtraction).
    \item
        flat fielding. Note problem that dome flats have spectral
        energy distribution of light source. ``Flatten'' the flats in
        the wavelength direction to preserve error analysis, i.e.\
        remove the large scale wavelength dependence, but preserve the
        pixel-to-pixel response variations. In the spatial direction,
        flat fielding is like imaging, but often the requirements on
        accuracy are less stringent. An extra spatial component in the
        flats comes from variation of slit width.
    \item wavelength calibration. Use arc lamps with known lines.
        Identify lines, determine line centers (centroid or fitting),
        and fit function to centers vs.\ wavelength.
    \item flux calibration: correction for throughput as a function of
        wavelength. Not always required, e.g.\ if measuring strengths
        relative to nearby continuum. Spectrophotometric standards,
        e.g. Massey et al. ApJ 328, 315 (1988).
        If fluxing is performed, usually also want to correct for
        atmospheric extinction as a function of wavelength and
        airmass: use of mean extinction coefficients.
    \item Object reduction: extracting object spectrum (``tracing'' the
        object) and sky spectrum. Aperture extraction vs.\ optimal
        extraction. Caveats: spectral curvature.
    \item Advanced topics: nod and shuffle, atmospheric feature
        correction (esp in IR).
\end{enumerate*}

\subsection*{IRAF utilities}
\begin{itemize*}
    \item IRAF: \href{http://iraf.noao.edu/iraf/web/tutorials/doslit/doslit.html}
        {response and doslit}
    \item load specred package:
        \begin{verbatim}
iraf.imred()
iraf.specred()
        \end{verbatim}
    \item response takes out the observed flat field response in the
        wavelength direction (which is a combination of the flat field
        SED and the spectrograph response)
        doslit is the "meta" task that does wavelength calibration,
        flux calibration, and object extraction for point sources
        Images must be run through CCDPROC first (or have CCDPROC flag
        in header)
        For the arc list, beware that the .fits should not be included
        in the file name, it is automatically added (with imtype =
        fits)
\end{itemize*}


\end{document}
