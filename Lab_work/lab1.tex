\documentclass[12pt]{article}
\usepackage[margin=1in]{geometry}
\usepackage{fancyhdr}
\pagestyle{fancy}
\usepackage{enumerate}
\setlength{\parindent}{0em}
\setlength{\parskip}{0.5em}

\lhead{05 February 2016}
\chead{Lab 01}
\rhead{Laurel Farris}

\begin{document}

\begin{enumerate}[1.]
    \item I estimated the hourly increase in RA by dividing 365 days
    by 24 hours, which is about 0.0658 hours per day.
    Since the RA for an object crossing the meridian at midnight
    is 12 hr on March 21, I used the daily increase to calculate the
    RA for an object crossing the meridian at midnight on the first of
    each month. They are listed in table \ref{table:RA}.

        \begin{table}[h]
        \centering
        \begin{tabular}{c c}
        \hline\hline
        Month & RA at midnight\\
        \hline
        April & 12.72\\
        May & 14.70\\
        June & 16.73\\
        July & 18.71\\
        August & 20.75\\
        September & 22.78\\
        October & 0.76\\
        November & 2.79\\
        December & 4.77\\
        January & 6.81\\
        February & 8.84\\
        March & 10.68\\
        \hline
        \end{tabular}
        \caption{This table shows the approximate RA of an object
        crossing the meridian at midnight on the first day of each month.}
        \label{table:RA}
        \end{table}

    \item It is about 10am local time right now (solar time).
    Since the RA of an
    object crossing the meridian at midnight is about 8hrs, the RA of
    objects directly overhead right now is about 18 hrs, which is
    equal to the local sidereal time. The declination overhead is
    equal to our latitude, which is 32.3$^{\circ}$ North.
    Observing within 60$^{\circ}$ is ideal, so we would want to
    observe objects with an RA between 14 and 22 hours (60$^{\circ}$
    corresponds to four hours) and a declination between -28$^{\circ}$
    and 88$^{\circ}$ North (on the other side of the North
    Star$\ldots$ where the RA would actually be 12 hours from
    88$^{\circ}$ on the other side$\ldots$ I think.

    \item From \texttt{skycalendar} (If I'm reading it correctly),
        the nights during the first quarter (January, Februaray, and
        March) that have no moon above the horizon during the
        first half of the night are 1/31 - 2/5, 2/29 - 3/6, and 3/30 -
        3/31 (daylight savings starts 3/13, so the first-half night
        goes until 1am instead of midnight).

    \item  
        \begin{itemize}
            \item M87:
            RA = 12h 30m 49.42s;
            Dec = 12$^{\circ}$ 23$^{\prime}$ 28.04$^{\prime\prime}$
            \item Galactic center:
            RA = 17h 45.6m;
            Dec = -28.94$^{\circ}$
            \item Jupiter:
            RA = 12h 30m 49.42s;
            Dec = 12$^{\circ}$ 23$^{\prime}$ 28.04$^{\prime\prime}$
        \end{itemize}

    \item \textbf{Run skycalc (choose observatory A for APO, ? gives
    list of command help, look at r, d, y, and h commands). For the
    galactic center, what is the maximum amount of time it can be
    observed at an airmass of less than 2.5? How about the Virgo
    cluster? Why are these different?}

    \item \textbf{Run jskycalc. Play with all of the buttons! What
    planets will be visible spring 2016, and at what times of night?
    Note that you can load files with a list of coordinates, and you
    can make airmass observability charts for them.}

    \item \textbf{Start to outline plan for an 3 half-night observing
    run during late March A halves, when we are taking our APO trip.
    Eventually, the plan should include a list of objects for each
    night with a tentative order of observation, taking into account
    how much time needs to be spent on each object. Our projects are
    still TBD, but will likely include observations with multiple
    instruments.}
        \begin{itemize}
            \item \textbf{Determine the approximate range of RAs that
            we will be able to observe.}
            \item \textbf{Given the NMSU 1st quarter proposals, which
            of them might we be able to make some observations for?}
            \item \textbf{If you have other ideas for projects, start
            to tabulate them. (Sten/Diane stars for APOGEE
            calibration/neutron capture calibration, Triplespec RR
            Lyrae RV curves Drew Be stars)}
            \item \textbf{Start to prepare a joint web page with the
            plan, including relevant information: coordinates of
            objects, finder images if necessary, links to tabulated
            spectra, instrument manuals, etc. etc. }
        \end{itemize}

    \item \textbf{Look up the catalog Globular Clusters in the Milky
    Way in VizieR and download it (make sure to get all of the rows).}
        \begin{itemize}
            \item \textbf{Plot the locations in an Aitoff projection
            of equatorial coordinates. Can you detect Galactic
            structure?}
            \item \textbf{What clusters would be possible to observe
            during our March run? }
            \item \textbf{Convert coordinates to galactic coordinates
            and plot in an Aitoff projection.}
        \end{itemize}


\end{enumerate}


\end{document}
