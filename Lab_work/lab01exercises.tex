\documentclass[12pt]{article}
\usepackage[margin=1in]{geometry}
\usepackage{fancyhdr}
\pagestyle{fancy}
\usepackage{enumerate}
\setlength{\parindent}{0em}
\setlength{\parskip}{0.5em}

\lhead{05 February 2016}
\chead{Lab 01}
\rhead{Laurel Farris}

\begin{document}

\begin{enumerate}[1.]
    \item \textbf{Predict the RA crossing the meridian at midnight for the
    first of every month. Try the command skycalendar (on the cluster,
    unless you download it yourself for your laptop) - give yourself a
    wide terminal window first - to see how well you did}

    \item \textbf{What time is it now? What is the sidereal time? What
    coordinates would it be most optimal to observe right now?}

    \item \textbf{When are the dark (no moon above horizon) first half
    nights in first quarter ? }

    \item \textbf{APO schedules the 3.5m in half-night blocks (A and
    B), split at midnight (or 1am during daylight savings). What are
    the best half-nights in the next year (month and half, e.g., Oct
    A, March B, etc.) to request to observe:}
        \begin{itemize}
            \item \textbf{Virgo cluster of galaxies (note central
            galaxy is M87, look up the coordinates)}
            \item \textbf{Galactic center (galactic coordinates are
            .... ask if you don't know!). You can use command skycoor
            (or Python or IDL tools) to convert galactic to equatorial
            (skycoor with no arguments gives syntax). }
            \item \textbf{Jupiter (look up its position using JPL
            HORIZONS) }
        \end{itemize}

    \item \textbf{Run skycalc (choose observatory A for APO, ? gives
    list of command help, look at r, d, y, and h commands). For the
    galactic center, what is the maximum amount of time it can be
    observed at an airmass of less than 2.5? How about the Virgo
    cluster? Why are these different?}

    \item \textbf{Run jskycalc. Play with all of the buttons! What
    planets will be visible spring 2016, and at what times of night?
    Note that you can load files with a list of coordinates, and you
    can make airmass observability charts for them.}

    \item \textbf{Start to outline plan for an 3 half-night observing
    run during late March A halves, when we are taking our APO trip.
    Eventually, the plan should include a list of objects for each
    night with a tentative order of observation, taking into account
    how much time needs to be spent on each object. Our projects are
    still TBD, but will likely include observations with multiple
    instruments.}
        \begin{itemize}
            \item \textbf{Determine the approximate range of RAs that
            we will be able to observe. }
            \item \textbf{Given the NMSU 1st quarter proposals, which
            of them might we be able to make some observations for?}
            \item \textbf{If you have other ideas for projects, start
            to tabulate them. (Sten/Diane stars for APOGEE
            calibration/neutron capture calibration, Triplespec RR
            Lyrae RV curves Drew Be stars)}
            \item \textbf{Start to prepare a joint web page with the
            plan, including relevant information: coordinates of
            objects, finder images if necessary, links to tabulated
            spectra, instrument manuals, etc. etc. }
        \end{itemize}

    \item \textbf{Look up the catalog Globular Clusters in the Milky
    Way in VizieR and download it (make sure to get all of the rows).}
        \begin{itemize}
            \item \textbf{Plot the locations in an Aitoff projection
            of equatorial coordinates. Can you detect Galactic
            structure?}
            \item \textbf{What clusters would be possible to observe
            during our March run? }
            \item \textbf{Convert coordinates to galactic coordinates
            and plot in an Aitoff projection.}
        \end{itemize}


\end{enumerate}


\end{document}
