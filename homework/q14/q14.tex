\documentclass[12pt]{article}
\usepackage[margin=1in]{geometry}
\setlength{\parindent}{0em}
\setlength{\parskip}{0.5em}
\usepackage{fancyhdr}
\pagestyle{fancy}
\usepackage{color}
\usepackage{hyperref}

\lhead{Laurel Farris}
\chead{ASTR 535 - Homework 14}
\rhead{19 February 2016}

\begin{document}

\bibliographystyle{plain}

\textbf{Errors and error propagation}\small
\begin{enumerate}
    \item\textbf{If you observe a source that emits, on average, 10
    photons per second, and take a 10 second exposure, how many counts
    will you observe on average, and what will be expected standard
    deviation if you make a series of measurements? What is the
    signal-to-noise of each observation?}
    \item\textbf{If you combine multiple observations, you need to
    propogate the errors on each individual observation to get an
    error estimate on your combined quantity. If you average together
    10 of the 10s exposures, what is the expected error on this
    average?}
    \item\textbf{If you take a single observation and count 1000
    photons, what is the expected error? What is the expected error
    (in magnitudes) if you convert the counts into a magnitude?}
    \item\textbf{If you observe a source in the B passband and get 500
    photons, and observe it in the V passband and get 1000 photons,
    what is the expected error in the flux ratio, F(B)/F(V)? What is
    the expected error in the color difference (in magnitudes), mB -
    mV?}
    \item\textbf{In many situations, however, one receives photons not
    only from the astronomical source, but also ``background'' photons
    emitted by the night sky. To measure the source brightness, one
    needs to subtract the background, e.g. by measuring it separately
    from the source. This is not generally difficult. However, when
    initially counting photons from source and background combined,
    both of these contribute to the noise in the measurement.}
    \begin{enumerate}
        \item\textbf{If one observes an object that produces 1000
        photons in a ten minute exposure, and in that same time, the
        sky produces 500 photons, how many total photons will be
        measured? What is the expected error on this total from
        Poisson statistics?}
        \item\textbf{If you assume that you can subtract the sky
        perfectly (usually you can come close), then the only effect
        that the extra background has is in the increased noise. For
        the 1000 source photons + 500 background photons, what will be
        your expected signal to noise?}
        \item\textbf{When the noise from the object is the dominant
        noise source, this is called signal-limited. When the noise
        from the background is the dominant noise source, this is
        called background-limited. Derive how signal-to-noise depends
        on exposure time for both the signal-limited and
        background-limited cases.}
    \end{enumerate}

\end{enumerate}

\bibliography{reffile}

\end{document}
