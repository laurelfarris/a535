\documentclass[12pt]{article}
\usepackage{color}
\usepackage{hyperref} % \url{link} or \hrf{link}{text}
\usepackage{titlesec}
\titleformat{\section}%
  [hang]% <shape>
  {\normalfont\bfseries\Large}% <format>
  {}% <label>
  {0pt}% <sep>
  {}% <before code>
  \renewcommand{\thesection}{}% Remove section references...
  \renewcommand{\thesubsection}{\arabic{subsection}}%... from subsections
  \renewcommand{\thesubsubsection}{\arabic{subsubsection}}%... from subsections

\begin{document}

%\bibliographystyle{plain}

\section*{Homework 7}
\date{01/29/2016}
\subsection*{Astronomers use several different units to measure
brightnesses: fluxes per unit wavelength ($F_{\lambda}$),
fluxes per unit frequency ($F_{\nu}$), and magnitudes, both
integrated and per unit wavelength and frequency (m = -2.5logF).
Since magnitudes are logarithimic units, differences in magnitude
correspond to ratios in fluxes.}

\subsubsection{A Jansky is a unit used to measure flux density,
most often in the radio; one Jansky is 10$^{-26}$ W m$^{-2}$ Hz$^{-1}$
(note that this is an F$_{\nu}$ quantity).
How bright is Vega at 5500 \AA{} in Janskys, using the fact that the
flux density of Vega at 5500 \AA{}
is 3.6 $\times 10^{-9}$ erg cm$^{-2}$ s$^{-1}$ \AA{}$^{-1}$
(note that this is an F$_{\lambda}$ quantity)?}

\subsubsection{If a star has a flux density of 3.6 x
10-9ergs/cm2/sec/\AA{} at
8500 \AA{} how bright is it in Janskys?}

\subsubsection{If a star has a flux density of 7.2 x
10-14ergs/cm2/sec/\AA{} at 5500 \AA{}, how much fainter is it
than Vega in magnitudes? }


%\bibliography{reffile}

\end{document}
