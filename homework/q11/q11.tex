\documentclass[12pt]{article}
\usepackage[margin=1in]{geometry}
\setlength{\parindent}{0em}
\setlength{\parskip}{0.75em}
\usepackage{fancyhdr}
\pagestyle{fancy}

\rhead{10 February 2016}
\chead{ASTR 535 - Homework 11}
\lhead{Laurel Farris}

\begin{document}

\textbf{(2/10, 20 minutes) If you are looking at faint, background-limited,
stars, how much fainter can you see (i.e., measure at the same S/N) if
you have 0.3 arcsec instead of 1.0 arcsec images? Express your answer
in magnitudes.}

For background-limited stars, the signals and their corresponding
telescope sizes (diameters) are related as
    $$ S_2 = \sqrt{\frac{T_1}{T_2}}S_1 $$
    or
    $$ \frac{S_2}{S_1} = \sqrt{\frac{T_1}{T_2}} $$

Since the telescope size and resolution are inversely proportional
($ \theta = 1.22\lambda/T)$, a resolution that is 3$\times$ smaller
(at the same wavelength) would mean you have a telescope that is
3$\times$ bigger. 
\end{document}
