\documentclass[12pt]{article}
\usepackage[margin=1in]{geometry}
\usepackage{color}
\usepackage{hyperref} % \url{link} or \hrf{link}{text}
\usepackage{amsmath}
\usepackage{fancyhdr}
\pagestyle{fancy}
\pagenumbering{gobble}

\usepackage{titlesec}
\titleformat{\section}%
  [hang]% <shape>
  {\normalfont\bfseries\Large}% <format>
  {}% <label>
  {0pt}% <sep>
  {}% <before code>
  \renewcommand{\thesection}{}% Remove section references...
  \renewcommand{\thesubsection}{\arabic{subsection}}%... from subsections
  \renewcommand{\thesubsubsection}{\arabic{subsubsection}}%... from subsections


\lhead{Laurel Farris}
\chead{Homework 9}
\rhead{03 February 2016}

\begin{document}

\subsection*{(2/3, 15 minutes)
Imagine you are observing a binary star system,
but where the stars are so close together that you cannot
resolve them, and you see them as a single object.}

\subsubsection{If the two stars have magnitudes of 17 and 18
individually, what is the combined observed magnitude?}
First, convert each individual magnitude to the corresponding flux,
using the flux of Vega at 5500\AA{}
($F_{0,\lambda} = 3.60\times10^{-9}$ erg s$^{-1}$ cm$^{-2}$
\AA{}$^{-1}$)
as a reference flux:
    $$ m = -2.5\log\frac{F_{\lambda}}{F_{0,\lambda}}  $$
The flux of each star can then be added together to give the total flux
being received. This will be a monochromatic flux, but this shouldn't
make a difference once it is converted back to magnitude,
since magnitudes are nearly independent of bandpass.
The total flux converted back to magnitude gives the combined
magnitude from both stars:
    $$ m_{total} = -2.5\log\frac{F_{total}}{F_{0,\lambda}} $$

\subsubsection{Write a (short!) software function that takes
as input the apparent magnitude of each of two stars,
and computes and returns the apparent magnitude of the
two stars combined.}
The combined magnitude turns out to be about 16.64 mag
(see \texttt{q9.py}).
\end{document}
