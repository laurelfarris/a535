\documentclass[12pt]{article}
%\usepackage{cite}
\begin{document}

\bibliographystyle{plain}

\section*{Homework 3}
\subsubsection*{What is the solar constant (definition and value)?
    What is the surface brightness of the Sun as seen from Earth?
    What would the solar constant and surface brightness of the Sun
    be as seen from Jupiter?}

    The solar constant is the bolometric flux received from the sun at a
    distance of 1 astronomical unit (AU)\cite{kuhn}.
    The solar luminosity is 3.9$\times10^{33}$ erg s$^{-1}$, so at a distance of
    1.5$\times10^{13}$ cm, the solar constant is equal to
    1.73$\times10^7$ erg s$^{-1}$ cm$^{-2}$. The surface brightness of the
    sun as seen from Earth is

    From Jupiter, which is about 5 AU (7.5$\times10^{13}$ cm) from the Sun,
    the solar constant is about 6.93$\times10^5$ erg s$^{-1}$ cm$^{-2}$.
    The surface brightness of the Sun from Juipiter
    is the same as it is from Earth, as
    surface brightness is independent of distance at these scales.

\bibliography{reffile}

\end{document}
