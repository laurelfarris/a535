\documentclass[12pt]{article}
\usepackage[margin=1in]{geometry}
\usepackage{mdwlist}
\setlength{\parindent}{0em}
\setlength{\parskip}{0.5em}



\begin{document}

\subsection*{Overview}
The class will deliver an exposure time calculator for the 3.5m
telescope at Apache Point Observatory that will be tuned and validated
using observations taken during our class trip.

\subsection*{Deliverables}
The class will deliver either \textbf{one} or \textbf{two}
software products that
perform the required task. If two are delivered, they must be written
in different languages. \emph{However}, they must both be developed
collaboratively using, e.g., the same software design, e.g., names
and purposes of subroutines/functions; while they use the same
design, however, they should be coded independently (not translated)
so that they can serve to check one another. In other words, there
should not be two separate groups working on this, just one single
group with two implementations.

Each student will separately provide a brief description of their
individual contribution to the project, and their overall assessment
of the degree to which the collaborative work and the contributions
of all group members was effective. 

\subsection*{Performance Specifications}
The code will be able to accept as target input any of the following:
\begin{itemize*}
    \item the magnitude of an object along with the specified bandpass
        and an effective temperature (to give the shape of the spectrum)
    \item the flux of an object at a specific wavelength, and an
        effective temperature.
    \item the flux spectrum of an object 
\end{itemize*}
The code will accept as instrument configuration input:
\begin{itemize*}
    \item Moon phase (note that the effect of moonlight can be assumed
        for this purpose as a function of moon phase only)
    \item seeing
    \item airmass of observation
\end{itemize*}
The code will accept as choices for output:
\begin{itemize*}
    \item the required exposure time (accurate to 10\%) to reach a
        specified S/N.
    \item the S/N achived at a specified exposure time.
    \item the calculated flux for a specified exposure time
                (will be needed to tune the calculator).
\end{itemize*}
For all input, the code will provide reasonable default choices to
the user.

It should be easy for the user to provide input. A graphical
interface would probably be nice.

The code will produce the desired output for the given instrument
configuration;
\begin{itemize*}
    \item a set of numbers for a set of filters for the imaging
        instruments, or numbers as a function of wavelength for the
        spectrographs. Graphical output is probably the best along with a
        small table, e.g., for the filters and for a few specified
        wavelengths for the spectrographs.
    \item information providing the relative noise contributions of the
        different noise sources: Poisson statistics from object,
        Poisson statistics from sky, and readout noise.
\end{itemize*}

\subsection*{Code Specifications}
The code should be designed to minimize the number of required lines.
This will require careful advance thought and planning.

The code should be written so that is is flexible, e.g., so that it
could easy be ported for use with another telescope/instrument suite.

The code should be maximally readable and modular. All
procedures/subroutines/functions should have a brief description of
their function, and a brief description of input and output
quantitities.

It should be able to be printed with 80 characters per line without
line wrap.

Python code should conform to PEP-8 specifications.

IDL code should use the DOC\_LIBRARY standard format for documenting
procedures/functions.

Comments should be complete but concise; over-commenting should be
avoided. In-line comments should be used minimally, if at all.
Documentation (see below) can be more extensive.

White space should be used appropriately, but judiciously; compact
code is a plus. 

\subsection*{Documentation specifications}
The code should be delivered with documentation for how it should be
used and how it works. It is strongly encouraged that such
documentation be done using a code documentation system such as Sphinx
or doxygen.

\subsection*{Personnel specifications}
Workload should be divided among students, with some clear written
identification of roles and responsibilities. Everyone should
participate in code design and planning. Different people could take
on different other roles, for example: information collector (details
about instruments; mirror, instrument, filter, detector efficiencies
as a function of wavelength), programmer, tester, documentation, etc.

\subsection*{Information collection}
 You will need to get estimates for telescope, instrument, and
 detector efficiencies. You may be able to get much of this from the
 web; beyond this, you should consult with Jon Holtzman as your
 primary source (he can point you to other people/resources if
 needed).

Note that there will probably be some empirical correction required
to make your calculations sufficiently accurate. At some level, an
exposure time calculator can be used as an instrument throughput
indicator, and you should think about what observations you might
require during our class trip to derive any needed empirical
corrections. The code should be written anticipating the
incorporation of such corrections. 

\subsection*{Timeline}
The tool should be complete by the last day of classes. The initial
planning should be complete by spring break. Some initial code
development probably needs to occur before the observing run 3/25-27,
so that you can collect required information at the telescope to
refine it (note that you have all written initial attempts at this,
so much of what you need may already be done in some form).

The group will provide a weekly (every Friday) brief report on
progress made during the preceeding week.

\end{document}






















