\documentclass[12pt]{article}
\usepackage[margin=1in]{geometry}
\setlength{\parindent}{0em}
\setlength{\parskip}{1.0em}
\usepackage{fancyhdr}
\pagestyle{fancy}
\usepackage{color}
\usepackage{hyperref}

\lhead{Laurel Farris}
\chead{ASTR 535 - Homework 13}
\rhead{12 February 2016}

\begin{document}

\textbf{(2/12, 1 hour) Continue working on your exposure time
calculator software by adding routines to calculate the S/N of a
given exposure. You will want to call your count equation routine
to get the signal for an object of some input magnitude, call it
again to get the background for a given sky brightness, allow for
an input sky area/image size (with suitable default value), and
call a routine to return the readout noise for an input pixel
scale and input sky area/image size. Validate your routine using
the results from the previous question.}

\textbf{(Previous question) Start to develop a software module that
implements the count equation that will eventually be used for an
exposure time calculator. The module should include individual
functions that return values for each of the terms in the count
equation: photon flux given input magnitude, telescope area,
atmospheric transmission, system transmission (split into separate
routines for each of telescope throughput, instrument throughput,
filter throughput, and detector efficiency). Each routine should
accept as input a wavelength or array of wavelengths, and return
values at all of the specified input wavelengths; for the initial
effort you could just return a constant value at all wavelengths, but
look forward to building in the capability for wavelength dependence,
e.g., interpolating from a list of wavelenth/values from an input
file. A higher level function should return the estimated counts
integrated over all wavelengths. You should make appropriate use of
input keywords and default values. You are strongly encouraged to
sketch out the design of the program before starting to write it,
considering the eventual use of the program to provide exposure time
estimates for all of the APO 3.5m instruments (and perhaps others),
where you will have wavelength dependent throughputs for multiple
components. Test the program by using it to do the calculation for the
NMSU 1m.}

\end{document}
