\documentclass[12pt]{article}
\usepackage[margin=1in]{geometry}
\usepackage{enumerate}
\setlength{\parindent}{0em}
\setlength{\parskip}{0.75em}
\usepackage{fancyhdr}
\pagestyle{fancy}

\lhead{Laurel Farris}
\chead{Homework 10}
\rhead{05 February 2016}



\begin{document}

\section*{\normalsize{Count equation and exposure time calculator.}}

\begin{enumerate}[1.]
    \item (2/5, 2 hours) Estimate the number of photons/second
you would receive for a star with V=22 with the NMSU 1m assuming that
the combined detection efficiency of the telescope and detector is 50
percent. You can assume that the V bandpass is rectangular with a
central wavelength of 5500\AA{}, a full width of 1000\AA{}, with an
in-band transmission of 80 percent.
    \item something
\end{enumerate}

Here is my answer to the question.

\end{document}
