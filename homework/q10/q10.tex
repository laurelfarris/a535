\documentclass[12pt]{article}
\usepackage[margin=1in]{geometry}
\usepackage{enumerate}
\setlength{\parindent}{0em}
\setlength{\parskip}{0.5em}
\usepackage{fancyhdr}
\pagestyle{fancy}

\lhead{Laurel Farris}
\chead{Homework 10}
\rhead{05 February 2016}

\usepackage{titlesec}
\titleformat{\section}%
[hang]% <shape>
{\normalfont\bfseries\Large}% <format>
{}% <label>
{0pt}% <sep>
{}% <before code>
\renewcommand{\thesection}{}
\renewcommand{\thesubsection}{\arabic{subsection}}
\renewcommand{\thesubsubsection}{\arabic{subsubsection}}

\begin{document}

\subsection*{Count equation and exposure time calculator.}

\subsubsection{(2/5, 2 hours) Estimate the number of photons/second
you would receive for a star with V=22 with the NMSU 1m assuming that
the combined detection efficiency of the telescope and detector is 50
percent. You can assume that the V bandpass is rectangular with a
central wavelength of 5500\AA{}, a full width of 1000\AA{}, with an
in-band transmission of 80 percent.}
The number of photons received would be about 2 billion per second
(see \texttt{q10.py}).
\subsubsection{What is the zeropoint for the V bandpass for the 1m,
        using the approximate:
        $$ \textrm{standard\ mag} = - 2.5\times\log(\textrm{photons}/s)
        + {\textrm{zeropoint}} $$}
        The zeropoint is about 21.11
        (see \texttt{q10.py}).
\subsubsection{Start to develop a software module that implements the count
        equation that will eventually be used for an exposure time calculator.
        The module should include individual functions that return values for
        each of the terms in the count equation: photon flux given input magnitude,
        telescope area, atmospheric transmission, system transmission
        (split into separate routines for each of telescope throughput,
        instrument throughput, filter throughput, and detector efficiency).
        Each routine should accept as input a wavelength or array of wavelengths,
        and return values at all of the specified input wavelengths;
        for the initial effort you could just return a constant value
        at all wavelengths, but look forward to building in the capability
        for wavelength dependence, e.g., interpolating from a list of
        wavelenth/values from an input file. A higher level function should
        return the estimated counts integrated over all wavelengths.
        You should make appropriate use of input keywords and default values.
        You are strongly encouraged to sketch out the design of the program
        before starting to write it, considering the eventual use of the
        program to provide exposure time estimates for all of the APO 3.5m
        instruments (and perhaps others), where you will have wavelength
        dependent throughputs for multiple components.
        Test the program by using it to do the calculation for the
        NMSU 1m.}

        See \texttt{exposure\_time\_calc.py} in the current directory.


\end{document}
