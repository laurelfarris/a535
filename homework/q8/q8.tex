\documentclass[12pt]{article}
\usepackage{color}
\usepackage{hyperref} % \url{link} or \hrf{link}{text}
\usepackage{titlesec}
\usepackage{amsmath}
\titleformat{\section}%
  [hang]% <shape>
  {\normalfont\bfseries\Large}% <format>
  {}% <label>
  {0pt}% <sep>
  {}% <before code>
  \renewcommand{\thesection}{}% Remove section references...
  \renewcommand{\thesubsection}{\arabic{subsection}}%... from subsections
  \renewcommand{\thesubsubsection}{\arabic{subsubsection}}%... from subsections

\addtolength{\oddsidemargin}{-0.875in}
\addtolength{\evensidemargin}{-0.875in}
\addtolength{\textwidth}{1.75in}
\addtolength{\topmargin}{-0.875in}
\addtolength{\textheight}{1.75in}

\begin{document}

\section*{Homework 8}
\date{01/29/2016}
\subsection*{
(1/29, 30 minutes)A color of a star gives the ratio of fluxes in two
different wavelength bandpasses. When expressed in magnitude units,
colors are given by a difference in magnitude between two different
wavelength bandpasses and are often defined relative to the colors of
some other objects. For example, in the VEGAMAG (e.g., UBVRI) system,
all colors are relative to the color of the star Vega, i.e. in this
system, the color of Vega is defined to be zero. The spectrum
(F$\scriptstyle \lambda$) of Vega is (very!) roughly proportional to $
\lambda^{{-2}}_{}$ (in the optical), i.e. F$\scriptstyle
\lambda$(Vega) $ \approx$ 3.6$\times10^{-9} \left(\vphantom{{\lambda\over
5500}}\right.$$ {\lambda\over 5500}$$ \left.\vphantom{{\lambda\over
5500}}\right)^{{-2}}_{}$erg cm$^{-2}$ s$^{-1}$ \AA{}$^{-1}$.}

\subsubsection{
If you approximate the B bandpass as a square bandpass between 4000
and 5000 \AA{}, and the V bandpass as a square bandpass between 5000 and
6000 \AA{}, what is the flux of Vega in both B and V}

\begin{align*}
    F &= \int F_{\lambda}\textrm{d}\lambda\\
    F_B &= \int_{4000}^{5000} 3.6\times10^{-9}
    \Big(\frac{\lambda}{5500}\Big)^{-2}\lambda\textrm{d}\lambda
    = \sim2.43\times10^{-18}\\
    F_V &= \int_{5000}^{6000} 3.6\times10^{-9}
    \Big(\frac{\lambda}{5500}\Big)^{-2}\lambda\textrm{d}\lambda
    = \sim1.99\times10^{-18}\\
\end{align*}
so $F_B$ and $F_V$ are about $2.43\times10^{-18}$ and $1.99\times10^{-18}$ 
erg s$^{-1}$ cm$^{-2}$, respectively.

\subsubsection{
If you observe a star that has B-V = 1 and V = 20, what is its flux
in the B and V bandpasses?}

The flux of a star of magnitude
$m_V$ can be calculated using the relation:
\begin{align*}
    m_V &= -2.5 \log \frac{F_V}{F_{V,0}}\\
\end{align*}
The flux of Vega ($F_{V,0}$) in the V bandpass is about 3.63$\times10^{-9}$
erg s$^{-1}$ cm$^{-2}$ \AA{}$^{-1}$.
For a star with $m_V$=20:
\begin{align*}
    20 &= -2.5 \log \frac{F_V}{3.63\times10^{-9}}\\
    F_V &= [10^{\frac{20}{-2.5}}]3.63\times10^{-9}\\
    &= \sim6.63\times10^{-17}
\end{align*}
Since B-V = 1, $F_B$ for the star can be calculated using the
relation:
\begin{align*}
    B-V &= -2.5 \log \frac{F_B}{F_V}\\
    1 &= -2.5 \log \frac{F_B}{6.63\times10^{-17}}\\
    F_B &= [10^{\frac{1}{-2.5}}]6.63\times10^{-17}\\
    &= \sim2.64\times10^{-17}
\end{align*}

\subsubsection{
What does it mean, quantitatively, for an object to have B-V = 1.0 ?}

The object's B-magnitude is higher than its V-magnitude, so it is a
cooler (redder) star (relative to an A0V star, like Vega).
This difference in magnitudes
corresponds to a flux ratio of about 2.51, so the star is about two
and a half times brighter through the V filter than it is through the
B filter.

\subsubsection{
If a star with a spectral energy distribution like Vega has a
magnitude of 21 at 4500 \AA{} in the VEGAMAG system (i.e. spectrum same
shape as Vega, but normalized differently), what would its magnitude
(m(4500)) be in the STMAG system? the ABNU system? What would the
m(4500) - m(5500) color be in each of the three systems?
}

\begin{align*}
    m_{\lambda=4500} &= -2.5 \log \frac{F_{\lambda=4500}}
    {F_{\lambda=4500,0}}\\
    21 &= -2.5\log\frac{F_{\lambda=4500}}{3.6\times10^{-9}
    \Big(\frac{4500}{5500}\Big)^{-2}}\\
    F_{\lambda=4500} &= 2.14\times10^{-17}
\end{align*}

\end{document}
