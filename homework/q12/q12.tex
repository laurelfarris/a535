\documentclass[12pt]{article}
\usepackage[margin=1in]{geometry}
\setlength{\parindent}{0em}
\setlength{\parskip}{0.75em}
\usepackage{fancyhdr}
\pagestyle{fancy}
\usepackage{amsmath}

\rhead{10 February 2016}
\chead{ASTR 535 - Homework 12}
\lhead{Laurel Farris}

\begin{document}

\textbf{(2/10, 30 minutes) Say you have a system that has a
photometric zeropoint of 23.5 in the bandpass where you plan to make
observations, and you are observing at a site with 1 arcsec seeing, a
background of 21 magnitudes per square arcsec with a detector with
readout noise corresponding to 5 electrons/pixel, and a plate scale of
0.5 arcsec/pixel.}
\begin{enumerate}
    \item \textbf{How long do you need to expose to get a S/N of 100 for a
        15th magnitude object?}

    The signal-to-noise ratio ($S/N$) is given by
    $$ \frac{S}{N} = \frac{S}{\sqrt{S+AB+N_{pix}\sigma_{rn}^2}} $$
    where $B$ = $B'Tt$. With the zeropoint, z, the background can be
    converted from magnitudes to counts, using
    $$ m = -2.5\log\frac{counts}{second} + z $$
    The background was given in magnitudes per square arcsecond, so
    this is $B'T$. Rearranging, we have
    $$ B'T = 10^{\frac{21-23.5}{-2.5}} $$
    $$ B'T = 10 $$
    Since the seeing is 1 arsec, and a single pixel element covers 0.25
    arsec$^2$, we need four pixels to cover one resolution element.
    So $N_{pix}$ = 4, and $A$ = 1.


\item \textbf{How long do you need to expose to get a S/N of 50 for a 22nd
    magnitude object?}
\end{enumerate}

\end{document}
