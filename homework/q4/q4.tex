\documentclass[12pt]{article}
\usepackage{color}
\usepackage{amsmath}
\usepackage{hyperref} % \url{link} or \hrf{link}{text}
\usepackage{titlesec}
\titleformat{\section}%
  [hang]% <shape>
  {\normalfont\bfseries\Large}% <format>
  {}% <label>
  {0pt}% <sep>
  {}% <before code>
  \renewcommand{\thesection}{}% Remove section references...
  \renewcommand{\thesubsection}{\arabic{subsection}}%... from subsections
  \renewcommand{\thesubsubsection}{\arabic{subsubsection}}%... from subsections

\begin{document}

\section*{Homework 4}
\subsection*{Magnitudes}
\subsubsection{Two objects from the SDSS survey have g=17 and g=19.5,
respectively. Which is brighter, and by how much?}

The difference in magnitude between the two stars is 2.5, which corresponds
to a brightness difference of a factor of 10, using the relationship
\begin{equation*}
    m_1 - m_2 = -2.5 \log \Bigg(\frac{b_1}{b_2}\Bigg)
\end{equation*}
Since objects with numerically higher magnitude values are fainter,
the object with g=17 is 10 times brighter than the object with g=19.5.

\subsubsection{Star A has V=12, star B has V=14.5. Which is brighter,
and by how much?}

Using the same reasoning as in question 1,
star A is 10 times brighter than star B.

\subsubsection{A star has B-V=0. What does that imply about the slope of its
spectrum? Is the slope independent of whether $F_{\nu}$ or $F_{\lambda}$
is being plotted? If not, how do they differ?}

If B-V=0, then B=V. In other words, the magnitude in the `blue' part of
the spectrum is the same as the magnitude in the `green', or visual
part of the spectrum. Since the UBV system is a broad-band photometric
system, B and V both cover a relatively wide range of wavelengths, so
while the $F_{\lambda}$ spectrum may be flat, the $F_{\nu}$ won't
necessarily be.

\end{document}
